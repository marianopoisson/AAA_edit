
%%%%%%%%%%%%%%%%%%%%%%%%%%%%%%%%%%%%%%%%%%%%%%%%%%%%%%%%%%%%%%%%%%%%%%%%%%%%%%
%  ************************** AVISO IMPORTANTE **************************    %
%                                                                            %
% Éste es un documento de ayuda para los autores que deseen enviar           %
% trabajos para su consideración en el Boletín de la Asociación Argentina    %
% de Astronomía.                                                             %
%                                                                            %
% Los comentarios en este archivo contienen instrucciones sobre el formato   %
% obligatorio del mismo, que complementan los instructivos web y PDF.        %
% Por favor léalos.                                                          %
%                                                                            %
%  -No borre los comentarios en este archivo.                                %
%  -No puede usarse \newcommand o definiciones personalizadas.               %
%  -SiGMa no acepta artículos con errores de compilación. Antes de enviarlo  %
%   asegúrese que los cuatro pasos de compilación (pdflatex/bibtex/pdflatex/ %
%   pdflatex) no arrojan errores en su terminal. Esta es la causa más        %
%   frecuente de errores de envío. Los mensajes de "warning" en cambio son   %
%   en principio ignorados por SiGMa.                                        %
%                                                                            %
%%%%%%%%%%%%%%%%%%%%%%%%%%%%%%%%%%%%%%%%%%%%%%%%%%%%%%%%%%%%%%%%%%%%%%%%%%%%%%

%%%%%%%%%%%%%%%%%%%%%%%%%%%%%%%%%%%%%%%%%%%%%%%%%%%%%%%%%%%%%%%%%%%%%%%%%%%%%%
%  ************************** IMPORTANT NOTE ******************************  %
%                                                                            %
%  This is a help file for authors who are preparing manuscripts to be       %
%  considered for publication in the Boletín de la Asociación Argentina      %
%  de Astronomía.                                                            %
%                                                                            %
%  The comments in this file give instructions about the manuscripts'        %
%  mandatory format, complementing the instructions distributed in the BAAA  %
%  web and in PDF. Please read them carefully                                %
%                                                                            %
%  -Do not delete the comments in this file.                                 %
%  -Using \newcommand or custom definitions is not allowed.                  %
%  -SiGMa does not accept articles with compilation errors. Before submission%
%   make sure the four compilation steps (pdflatex/bibtex/pdflatex/pdflatex) %
%   do not produce errors in your terminal. This is the most frequent cause  %
%   of submission failure. "Warning" messsages are in principle bypassed     %
%   by SiGMa.                                                                %
%                                                                            % 
%%%%%%%%%%%%%%%%%%%%%%%%%%%%%%%%%%%%%%%%%%%%%%%%%%%%%%%%%%%%%%%%%%%%%%%%%%%%%%

\documentclass[baaa]{baaa}

%%%%%%%%%%%%%%%%%%%%%%%%%%%%%%%%%%%%%%%%%%%%%%%%%%%%%%%%%%%%%%%%%%%%%%%%%%%%%%
%  ******************** Paquetes Latex / Latex Packages *******************  %
%                                                                            %
%  -Por favor NO MODIFIQUE estos comandos.                                   %
%  -Si su editor de texto no codifica en UTF8, modifique el paquete          %
%  'inputenc'.                                                               %
%                                                                            %
%  -Please DO NOT CHANGE these commands.                                     %
%  -If your text editor does not encodes in UTF8, please change the          %
%  'inputec' package                                                         %
%%%%%%%%%%%%%%%%%%%%%%%%%%%%%%%%%%%%%%%%%%%%%%%%%%%%%%%%%%%%%%%%%%%%%%%%%%%%%%
 
\usepackage[pdftex]{hyperref}
\usepackage{subfigure}
\usepackage{natbib}
\usepackage{helvet,soul}
\usepackage[font=small]{caption}

%%%%%%%%%%%%%%%%%%%%%%%%%%%%%%%%%%%%%%%%%%%%%%%%%%%%%%%%%%%%%%%%%%%%%%%%%%%%%%
%  *************************** Idioma / Language **************************  %
%                                                                            %
%  -Ver en la sección 3 "Idioma" para mas información                        %
%  -Seleccione el idioma de su contribución (opción numérica).               %
%  -Todas las partes del documento (titulo, texto, figuras, tablas, etc.)    %
%   DEBEN estar en el mismo idioma.                                          %
%                                                                            %
%  -Select the language of your contribution (numeric option)                %
%  -All parts of the document (title, text, figures, tables, etc.) MUST  be  %
%   in the same language.                                                    %
%                                                                            %
%  0: Castellano / Spanish                                                   %
%  1: Inglés / English                                                       %
%%%%%%%%%%%%%%%%%%%%%%%%%%%%%%%%%%%%%%%%%%%%%%%%%%%%%%%%%%%%%%%%%%%%%%%%%%%%%%

\contriblanguage{1}

%%%%%%%%%%%%%%%%%%%%%%%%%%%%%%%%%%%%%%%%%%%%%%%%%%%%%%%%%%%%%%%%%%%%%%%%%%%%%%
%  *************** Tipo de contribución / Contribution type ***************  %
%                                                                            %
%  -Seleccione el tipo de contribución solicitada (opción numérica).         %
%                                                                            %
%  -Select the requested contribution type (numeric option)                  %
%                                                                            %
%  1: Artículo de investigación / Research article                           %
%  2: Artículo de revisión invitado / Invited review                         %
%  3: Mesa redonda / Round table                                             %
%  4: Artículo invitado  Premio Varsavsky / Invited report Varsavsky Prize   %
%  5: Artículo invitado Premio Sahade / Invited report Sahade Prize          %
%  6: Artículo invitado Premio Sérsic / Invited report Sérsic Prize          %
%%%%%%%%%%%%%%%%%%%%%%%%%%%%%%%%%%%%%%%%%%%%%%%%%%%%%%%%%%%%%%%%%%%%%%%%%%%%%%

\contribtype{1}

%%%%%%%%%%%%%%%%%%%%%%%%%%%%%%%%%%%%%%%%%%%%%%%%%%%%%%%%%%%%%%%%%%%%%%%%%%%%%%
%  ********************* Área temática / Subject area *********************  %
%                                                                            %
%  -Seleccione el área temática de su contribución (opción numérica).        %
%                                                                            %
%  -Select the subject area of your contribution (numeric option)            %
%                                                                            %
%  1 : SH    - Sol y Heliosfera / Sun and Heliosphere                        %
%  2 : SSE   - Sistema Solar y Extrasolares  / Solar and Extrasolar Systems  %
%  3 : AE    - Astrofísica Estelar / Stellar Astrophysics                    %
%  4 : SE    - Sistemas Estelares / Stellar Systems                          %
%  5 : MI    - Medio Interestelar / Interstellar Medium                      %
%  6 : EG    - Estructura Galáctica / Galactic Structure                     %
%  7 : AEC   - Astrofísica Extragaláctica y Cosmología /                      %
%              Extragalactic Astrophysics and Cosmology                      %
%  8 : OCPAE - Objetos Compactos y Procesos de Altas Energías /              %
%              Compact Objetcs and High-Energy Processes                     %
%  9 : ICSA  - Instrumentación y Caracterización de Sitios Astronómicos
%              Instrumentation and Astronomical Site Characterization        %
% 10 : AGE   - Astrometría y Geodesia Espacial
% 11 : ASOC  - Astronomía y Sociedad                                             %
% 12 : O     - Otros
%
%%%%%%%%%%%%%%%%%%%%%%%%%%%%%%%%%%%%%%%%%%%%%%%%%%%%%%%%%%%%%%%%%%%%%%%%%%%%%%

\thematicarea{1}

%%%%%%%%%%%%%%%%%%%%%%%%%%%%%%%%%%%%%%%%%%%%%%%%%%%%%%%%%%%%%%%%%%%%%%%%%%%%%%
%  *************************** Título / Title *****************************  %
%                                                                            %
%  -DEBE estar en minúsculas (salvo la primer letra) y ser conciso.          %
%  -Para dividir un título largo en más líneas, utilizar el corte            %
%   de línea (\\).                                                           %
%                                                                            %
%  -It MUST NOT be capitalized (except for the first letter) and be concise. %
%  -In order to split a long title across two or more lines,                 %
%   please use linebreaks (\\).                                              %
%%%%%%%%%%%%%%%%%%%%%%%%%%%%%%%%%%%%%%%%%%%%%%%%%%%%%%%%%%%%%%%%%%%%%%%%%%%%%%
% Dates
% Only for editors
\received{\ldots}
\accepted{\ldots}




%%%%%%%%%%%%%%%%%%%%%%%%%%%%%%%%%%%%%%%%%%%%%%%%%%%%%%%%%%%%%%%%%%%%%%%%%%%%%%



\title{Model-based identification of solar coronal mass ejections using deep neural networks}

%%%%%%%%%%%%%%%%%%%%%%%%%%%%%%%%%%%%%%%%%%%%%%%%%%%%%%%%%%%%%%%%%%%%%%%%%%%%%%
%  ******************* Título encabezado / Running title ******************  %
%                                                                            %
%  -Seleccione un título corto para el encabezado de las páginas pares.      %
%                                                                            %
%  -Select a short title to appear in the header of even pages.              %
%%%%%%%%%%%%%%%%%%%%%%%%%%%%%%%%%%%%%%%%%%%%%%%%%%%%%%%%%%%%%%%%%%%%%%%%%%%%%%

\titlerunning{Model-based identification of CMEs using DNN}

%%%%%%%%%%%%%%%%%%%%%%%%%%%%%%%%%%%%%%%%%%%%%%%%%%%%%%%%%%%%%%%%%%%%%%%%%%%%%%
%  ******************* Lista de autores / Authors list ********************  %
%                                                                            %
%  -Ver en la sección 3 "Autores" para mas información                       % 
%  -Los autores DEBEN estar separados por comas, excepto el último que       %
%   se separar con \&.                                                       %
%  -El formato de DEBE ser: S.W. Hawking (iniciales luego apellidos, sin     %
%   comas ni espacios entre las iniciales).                                  %
%                                                                            %
%  -Authors MUST be separated by commas, except the last one that is         %
%   separated using \&.                                                      %
%  -The format MUST be: S.W. Hawking (initials followed by family name,      %
%   avoid commas and blanks between initials).                               %
%%%%%%%%%%%%%%%%%%%%%%%%%%%%%%%%%%%%%%%%%%%%%%%%%%%%%%%%%%%%%%%%%%%%%%%%%%%%%%

\author{
F.A. Iglesias\inst{1,2},
F. Cisterna\inst{1},
M. Sanchez\inst{1},
Y. Machuca\inst{1},
D. Lloveras\inst{1,2},
F. Manini\inst{1,2},
F.M. Lopez\inst{1,2},
H. Cremades\inst{1,2}
\&
A. Asensio-Ramos\inst{3}
}

\authorrunning{Iglesias et. al.}

%%%%%%%%%%%%%%%%%%%%%%%%%%%%%%%%%%%%%%%%%%%%%%%%%%%%%%%%%%%%%%%%%%%%%%%%%%%%%%
%  **************** E-mail de contacto / Contact e-mail *******************  %
%                                                                            %
%  -Por favor provea UNA ÚNICA dirección de e-mail de contacto.              %
%                                                                            %
%  -Please provide A SINGLE contact e-mail address.                          %
%%%%%%%%%%%%%%%%%%%%%%%%%%%%%%%%%%%%%%%%%%%%%%%%%%%%%%%%%%%%%%%%%%%%%%%%%%%%%%

\contact{francisco.iglesias@um.edu.ar}

%%%%%%%%%%%%%%%%%%%%%%%%%%%%%%%%%%%%%%%%%%%%%%%%%%%%%%%%%%%%%%%%%%%%%%%%%%%%%%
%  ********************* Afiliaciones / Affiliations **********************  %
%                                                                            %
%  -La lista de afiliaciones debe seguir el formato especificado en la       %
%   sección 3.4 "Afiliaciones".                                              %
%                                                                            %
%  -The list of affiliations must comply with the format specified in        %          
%   section 3.4 "Afiliaciones".                                              %
%%%%%%%%%%%%%%%%%%%%%%%%%%%%%%%%%%%%%%%%%%%%%%%%%%%%%%%%%%%%%%%%%%%%%%%%%%%%%%


\institute{
Grupo de Estudios en Heliofísica de Mendoza, Universidad de Mendoza, Argentina 
\and
Consejo Nacional de Investigaciones Científicas y Técnicas, Argentina
 \and
Instituto de Astrofísica de Canarias, España 
% \and
% Departamento de Astrofísica 
}

%%%%%%%%%%%%%%%%%%%%%%%%%%%%%%%%%%%%%%%%%%%%%%%%%%%%%%%%%%%%%%%%%%%%%%%%%%%%%%
%  *************************** Resumen / Summary **************************  %
%                                                                            %
%  -Ver en la sección 3 "Resumen" para mas información                       %
%  -Debe estar escrito en castellano y en inglés.                            %
%  -Debe consistir de un solo párrafo con un máximo de 1500 (mil quinientos) %
%   caracteres, incluyendo espacios.                                         %
%                                                                            %
%  -Must be written in Spanish and in English.                               %
%  -Must consist of a single paragraph with a maximum  of 1500 (one thousand %
%   five hundred) characters, including spaces.                              %
%%%%%%%%%%%%%%%%%%%%%%%%%%%%%%%%%%%%%%%%%%%%%%%%%%%%%%%%%%%%%%%%%%%%%%%%%%%%%%

\resumen{Las eyecciones coronales de masa (ECM) son un factor determinante del clima espacial y, por lo tanto, pueden tener impactos tecnológicos y sociales negativos importantes. Para poder predecir su geoefectividad, es crucial su identificación en imágenes de coronógrafos. En la última década, las redes neuronales profundas (RNP) han experimentado enormes mejoras para resolver diversas tareas relacionadas con la visualización por computadora. Un problema al intentar utilizar RNP para la segmentación de una ECM, es que no existe un conjunto de datos grande y curado en la literatura que pueda utilizarse para el entrenamiento supervisado. Hemos creado un conjunto de datos sintético de imágenes de coronógrafos de ECM que incorpora las características principales de interés, combinando imágenes reales de coronógrafos, con ECM sintéticas obtenidas mediante el modelo geométrico Graduated Cylindrical Shell (GCS). Presentamos el entrenamiento y rendimiento preliminar de una RNP que permite identificar y segmentar la envoltura exterior de una ECM en imágenes de coronógrafos. La RNP se basa en un ajuste fino del modelo MaskR-CNN, y produce una máscara de segmentación similar a la del GCS de la ECM presente en una única imagen diferencial del coronágrafo. Comparamos nuestros resultados con los de otros dos algoritmos clásicos usados para segmentar ECM.}


\abstract{Coronal mass ejections (CMEs) are a major driver of space weather and thus can have important negative technological and social impacts. To assess their geoeffectiveness once they are ejected, it is crucial their prompt identification in coronagraph images. In the last decade, deep neural networks (DNN) have experienced enormous improvements in solving various machine-vision related tasks. One issue when trying to use DNN for CME segmentation, using coronagraph images, is that no large curated dataset exists in the literature that can be used for supervised training. We have produced a synthetic dataset of CME coronagraph images that incorporates the main features of interest, by combining actual quiet (no CME) coronagraph images with synthetic CMEs simulated using the Graduated Cylindrical Shell (GCS) geometric model. In this work, we present preliminary results of a DNN trained to identify and segment the outer envelope of CMEs. This is done by fine-tuning a pre-trained MaskR-CNN model, to produce a GCS-like mask of the CME, present in a single differential coronagraphic image. We compare our results with two other classic CME segmentation algorithms.
}

%%%%%%%%%%%%%%%%%%%%%%%%%%%%%%%%%%%%%%%%%%%%%%%%%%%%%%%%%%%%%%%%%%%%%%%%%%%%%%
%                                                                            %
%  Seleccione las palabras clave que describen su contribución. Las mismas   %
%  son obligatorias, y deben tomarse de la lista de la American Astronomical %
%  Society (AAS), que se encuentra en la página web indicada abajo.          %
%                                                                            %
%  Select the keywords that describe your contribution. They are mandatory,  %
%  and must be taken from the list of the American Astronomical Society      %
%  (AAS), which is available at the webpage quoted below.                    %
%                                                                            %
%  https://journals.aas.org/keywords-2013/                                   %
%                                                                            %
%%%%%%%%%%%%%%%%%%%%%%%%%%%%%%%%%%%%%%%%%%%%%%%%%%%%%%%%%%%%%%%%%%%%%%%%%%%%%%

\keywords{Sun: coronal mass ejections (CMEs) --- techniques: image processing --- methods: data analysis}

\begin{document}

\maketitle
\section{Introduction}\label{S_intro}
Given the current inability to forecast the occurrence of coronal mass ejections (CMEs), evaluating their geoeffectiveness is crucial. Accurate identification of CMEs in coronagraph images and the characterization of their morphology in three dimensions (3D) is essential for this task. However, up to date, CMEs have been imaged using coronagraphs located at only three different vantage points in the best case, e.g., those of the \textit{Solar and Heliospheric Observatory} (SoHO) and \textit{Solar Terrestrial Relations Observatory} (STEREO) space missions. To overcome this ill-posed problem, the Graduated Cylindrical Shell model \citep{Thernisien-etal2009} uses a simple croissant-like shell that depends only on six parameters. The GCS model is widely used in the literature to adjust the outer envelope of CMEs observed in differential coronagraph images, acquired simultaneously from two or three viewpoints. This procedure is almost exclusively done manually, by comparing only morphological aspects. This given that an automatic fit by minimizing the difference between model and measured brightness levels, is difficult because it requires: (a) modeling the rapidly changing CME internal density distribution and its associated brightness, and (b) identifying in the coronagraph images the sections (pixels) belonging to the CME structure and assessing their total brightness. Task (b) is particularly complicated because CMEs can be highly amorphous, and their brightness is noisy and faint compared to an also dynamic background, among other reasons. Moreover, when using differential images to reduce the background influence, different sections of the CME internal structure may appear at various differential brightness levels due to the inhomogeneous 3D velocity field of the CME internal material. This makes it very difficult or even impossible to identify the true CME outer shell predominant morphology. In this work, we explore deep neural networks to tackle the identification of CMEs using a GCS-based segmentation mask.

Automatic CME identification in coronagraph images and estimation of their basic morphological and kinematic parameters, such as angular width (AW), the central position angle (CPA), the apex distance, and velocity, have been approached using both manual and automated techniques. The manual identification relies on a trained operator, e.g., the Coordinated Data Analysis Workshop (CDAW) Large Angle and \textit{Spectrometric Coronagraph Experiment} (LASCO)
CME catalog, and is subjective and unsuitable for real-time applications. The automated approach groups various traditional and machine learning algorithms, which rely on identifying in a single image or a time series, the areas (pixels) belonging to the CME. Some examples of the automated approach are the Solar Eruptive Event Detection System (SEEDS) \citep{Olmedo-etal2008} and  Coronal SEgmentation Technique (CORSET) \citep{goussies2010} algorithms, and the CME Automatic detection and tracking with MachinE Learning (CAMEL)\citep{Wang2019} and Alshehhi\citep{Alshehhi2021} neural network-based methods. The automated methods are based on differential brightness changes and typically do not impose a strong constraint on the identified CME shape. This complicates modeling the identified region using a simply connected shape, like the one derived from parametric models such as the GCS. We propose a DNN-based model to segment individual CME images that produce GCS-like segmentation masks.

\section{Synthetic CMEs training dataset}
It consists of $1.5\times10^5$ synthetic images [515x512] pixels, their associated masks, labels, and bounding boxes for CME and occulter, see Fig.~\ref{fig:figure1}. The synthetic images are formed by adding: (1) a background corona randomly selected from a curated database of \textsl{STEREO}/\textsl{COR2} differential images with no CME present in the field of view, (2) a synthetic occulter according to the background image, (3) a synthetic CME differential brightness image, and (4) Gaussian noise. The image in (3) is obtained by applying ray tracing to the density distribution derived from the GCS model. Three model evaluations are used, the first (btot) is evaluated with random input parameters from a predefined range, the second (btot\textunderscore inner) is evaluated as btot except that the GCS height parameter is multiplied by a random scaling factor in the [0.55\,--\,0.85] range, the third (btot\textunderscore outer) is evaluated as btot\textunderscore inner but using a second scaling of 1.2. The relative intensities of all these components are randomized within a predefined range.

\begin{figure*}
  \centering
  \includegraphics[width=0.8\textwidth]{fig1.png}
  \caption{Synthetic GCS-based differential brightness coronagraph image \emph{(left)} and its associated binary masks for labels CME \emph{(center)} and occulter \emph{(right)}.}
  \label{fig:figure1}
\end{figure*}

\section{Deep neural network model for CME segmentation}
We use the \textsc{Mask R-CNN}  \citep{He2017} model, designed to perform semantic detection and segmentation of multiple objects in a single image. The architecture of this model is formed by combining a deep convolutional backbone for feature extraction, a \textsc{ResNet-50-FPN} (\url{www.pytorch.org/vision/main/models}), followed by a two-channel head. Channel one is a fully connected multilayer perceptron that outputs the object bounding-box location and its label. Channel two is a 2-to-4-layer convolutional network that outputs the object mask, i.e., one scalar score value per pixel within the bounding box that quantifies the probability of the pixel to belong to the object. The \textsc{Mask R-CNN} loss is a combination of errors in the inferred boxes, labels and masks.
We use fine tuning to train the model. This means that we first load the ~$4.6\times10^7$ \textsc{Mask R-CNN} parameters pretrained on the COCO (Common Objects in Context) image dataset (~$8\times10^5$ images, 100 classes). After that, we freeze the first two layers, i.e., they are not trained, before doing supervised training in 90\% of our synthetic CME dataset. We train for ~$1\times10^5$ batches of eigth images each, which are randomly rotated and normalized to the [0\,--\,1] range before feeding the model. We select the pixel threshold value of 0.5, which minimizes the model test loss (relative difference in the mask areas) on the remaining 10\% of the dataset.

\section{Preliminary Results}
For each input image, the trained model produces multiple masks of varied accuracies. This can be due to highly dynamic background structures, slow and/or faint CMEs, or the presence of multiple CMEs simultaneously, see Fig.~\ref{fig:figure2.1} and ~\ref{fig:figure3} . To select the appropriate mask for a given image, we do not rely solely on their score (as commonly done in other \textsc{Mask R-CNN} implementations) but use their morphological (AW and CPA) consistency among all the masks found in the images of a time series belonging to the same CME event, see Fig. ~\ref{fig:figure2.1}.

\begin{figure*}
  \centering
  \includegraphics[width=1\textwidth]{fig2.1.png}
  \caption{Masks found in real \textit{LASCO C2} differential images acquired on 24 April 2013 at 06:59 UT (a) and 07:23 UT (b).  We show only masks with score $>$ 0.25.  The final selected masks are presented in panels (c) and (d). The corresponding rectangular bounding boxes are computed from the masks using the Python function \textsc{cv2.boundingRect()}. }
  \label{fig:figure2.1}
\end{figure*}

\begin{figure*}
  \centering
  \includegraphics[width=0.7\textwidth]{fig3.png}
  \caption{\textit{SOHO}/\textit{LASCO} C2 24 April 2013 at 04:23 UT  showing the masks found for two CMEs }
  \label{fig:figure3}
\end{figure*}

CPA and AW for each mask are taken as the median and 95-to-5 percentiles difference of the mask pixel angular distribution around the occulter, respectively.  If no consistent mask is found for a given image, then no detection is reported. If there are less than two masks per CME event, the event is left with no detection.
As a first validation, we compare the event minimum AW and median CPA computed with our neural segmentation and the values given by the SEEDS and Vourlidas \citep{vourlidas2017} catalogs for 35 CMEs taken from the Coronal Mass Ejection Kinematic Database Catalogue (\textsc{ KINCAT}) catalog (www.affects-fp7.eu), see Fig.~\ref{fig:figure4}. We expect Vourlidas to be more accurate than SEEDS because events are manually selected, and the CME is masked using the CORSET algorithm. The CPA correlation is high with both catalogs; with the outliers belong to cases where SEEDS (see an example in Fig. ~\ref{fig:figure6}) or Vourlidas (see an example in Fig. ~\ref{fig:figure7}) detections failed. The AW correlation is much lower, with our estimations being typically larger than Vourlidas and lower than SEEDS (which is generally unreliable because it is pixel-based). An extra comparison with AW in the \textsc{KINCAT} catalog using GCS estimations will quantify AW estimation errors.

\begin{figure*}
  \centering
  \includegraphics[width=0.75\textwidth]{fig4.png}
  \caption{CPA (left) and AW (right) estimated with SEEDS and Vourlidas (see the legend) vs. our estimation (horizontal axis).  The outliers in CPA are shown in Figs. 5 to 8. }
  \label{fig:figure4}
\end{figure*}

\begin{figure}
  \centering
  \includegraphics[width=0.5\textwidth]{fig6.png}
  \caption{ SECCHI/\textit{COR2}-A  02 June 2008 at 06:37 UT. Our (left) and SEEDS (right) wrong detection (AW too large). }
  \label{fig:figure6}
\end{figure}

\begin{figure}
  \centering
  \includegraphics[width=0.5\textwidth]{fig7.png}
  \caption{ SECCHI/\textit{COR2}-A 04 August 2009 at 22:07 UT. Vourlidas (left) failed detection vs. ours (right). Vourdilas reports outflowing background material, which is likely the cause of the false detection.  }
  \label{fig:figure7}
\end{figure}

\section{Conclusions and future work}
Based on these preliminary results, we conclude:
\begin{itemize}
\item Our synthetic CME images do contain the main features that allow the neural network to segment the CME outer envelope using a simple, fully connected geometric mask in most of the cases. 
\item The MaskR-CNN model architecture, particularly the depth of the convolutional backbone, seems enough to capture the relevant image features to segment CMEs.
\item Basic morphological parameters such as CPA and AW derived from our GCS-based mask are in agreement with those derived from other methods. However, while the CPA correlation is high ($>85\,\%$), the dispersion of the derived AWs is more sensitive to the correct shape. 
\item Cases of wrong detections with our NN seem to be related to CMEs that present strong elements not included in the synthetic dataset, such as leading shock or deflected streamers. 
\end{itemize}
To further improve the segmentation quality we plan to:
\begin{itemize}
\item Increase dataset size and include in the synthetic images other important elements, such as the leading shock and deflected lateral streamers. 
\item Use a neural model that can ingest many time-instants simultaneously.
\item Compare the mask derived with our NN and the other algorithms with the masks derived by manual GCS fitting, considering the latter as a ground truth. This can help to devise wich method derives better CME properties such as the AW.
\end{itemize}

\begin{acknowledgement}
\texttt{FAI, MS, and FC are supported by The Max Planck Partner Group between the University of Mendoza and the Max Planck Institute for Solar System Research (MPS)}.
\end{acknowledgement}

% \section{Instrucciones}

% El BAAA admite dos categorías de contribución:
% \begin{itemize}
%     \item Breve (4 páginas), correspondiente a comunicación oral o mural.
%     \item Extensa (8 páginas), correspondiente a informe invitado, mesa redonda o premio.
% \end{itemize}
% El límite de páginas especificado para cada categoría aplica aún después de introducir correcciones arbitrales y editoriales. Queda a cargo de los autores hacer los ajustes de extensión que resulten necesarios. {No está permitido el uso de comandos que mo\-di\-fiquen las propiedades de espaciado y tamaño del texto}, tales como $\backslash${\tt small}, $\backslash${\tt scriptsize}, $\backslash${\tt vskip}, etc. 


% Tenga en cuenta los siguientes puntos para la correcta preparación de su manuscrito:
% \begin{itemize}
%     \item Utilice exclusivamente este macro ({\tt articulo-baaa65.tex}), no el de ediciones anteriores. El mismo puede ser descargado desde SiGMa o desde el sistema de edición en línea \href{https://www.overleaf.com/}{\emph{Overleaf}}, como la plantilla titulada  \emph{Boletín Asociación Argentina de Astronomía}.
%    \item Elaborar el archivo fuente (*.tex) de su contribución respetando el formato especificado en la Sec.~\ref{sec:guia}      
%    \item No está permitido el uso de definiciones o comandos personalizados en \LaTeX{}.
% \end{itemize}

% \subsection{Plazos de recepción de manuscritos}

% La recepción de trabajos correspondientes a comunicación oral o mural se extiende hasta el día {\bf 9 de febrero de 2024} inclusive. Las contribuciones tipo informe invitado, mesa redonda o premio, se recibirán hasta el {\bf 16 de febrero de 2024} inclusive. La recepción finalizará automáticamente en las fechas indicadas, por lo que no se admitirán contribuciones enviadas con fechas posteriores.


% \section{Guía de estilo para el BAAA}\label{sec:guia}

% Al elaborar su manuscrito, siga rigurosamente el estilo definido en esta sección. Esta lista no es exhaustiva, el manual de estilo completo está disponible en la sección \href{http://sigma.fcaglp.unlp.edu.ar/docs/SGM_docs_v01/Surf/index.html}{Instructivos} del SiGMa. Si algún caso no está incluido en el manual de estilo del BAAA, se solicita seguir el estilo de la revista Astronomy \& Astrophysics\footnote{\url{{https://www.aanda.org/for-authors/latex-issues/typography}}}.

% \subsection{Idioma del texto, resumen y figuras}

% El artículo puede escribirse en español o inglés a decisión del autor. El resumen debe escribirse siempre en ambos idiomas. Todas las partes del documento (título, texto, figuras, tablas, etc.)  deben estar en el idioma del texto principal. Al utilizar palabras de un lenguaje diferente al del texto (solo si es inevitable) incluirlas en {\em cursiva}.

% \subsection{Título}

% Inicie en letra mayúscula solo la primera palabra, nombres propios o acrónimos. Procure ser breve, de ser necesario divida el título en múltiples líneas, puede utilizar el corte de línea (\verb|\\|). No agregue punto final al título.

% \subsection{Autores}

% Los autores deben estar separados por comas, excepto el último que se separa con ``\verb|\&|''. El formato es: S.W. Hawking (iniciales luego apellidos, sin comas ni espacios entre las iniciales). Si envía varios artículos, por favor revisar que el nombre aparezca igual en todos ellos, especialmente en apellidos dobles y con guiones.

% \subsection{Afiliaciones}

% El archivo ({\sc ASCII}) {\tt BAAA\_afiliaciones.txt} incluido en este paquete, lista todas las afiliaciones de los autores de esta edición en el formato adoptado por el BAAA. En caso de no encontrar su institución, respete el formato: Instituto (Observatorio o Facultad), Dependencia institucional (para instituciones en Argentina sólo indique las siglas), País (en español).  No incluya punto final en las afiliaciones, excepto si es parte del nombre del país, como por ejemplo: ``EE.UU.".

% \subsection{Resumen}

% Debe consistir de un solo párrafo con un máximo de 1\,500 (mil quinientos) caracteres, incluyendo espacios. Debe estar escrito en castellano y en inglés. No están permitidas las referencias bibliográficas o imágenes. Evite el uso de acrónimos en el resumen. 

% \subsection{Palabras clave: \textit{Keywords}}

% Las palabras clave deben ser escritas en inglés y seleccionarse exclusivamente de la lista de la American Astronomical Society (AAS) \footnote{\url{https://journals.aas.org/keywords-2013/}}. Toda parte indicada entre paréntesis no debe incluirse. Por ejemplo, ``(stars:) binaries (including multiple): close'' debe darse como ``binaries: close''. Palabras que incluyen nombres individuales de objetos lo hacen entre paréntesis, como por ejemplo: ``galaxies: individual (M31)". Respete el uso de letras minúsculas y mayúsculas en el listado de la AAS. Note que el delimitador entre palabras clave es el triple guión. Las {\em keywords} de este artículo ejemplifican todos estos detalles. 

% Finalmente, además de las palabras clave listadas por la AAS, el BAAA incorpora a partir del Vol. 61B las siguientes opciones: {citizen science --- education --- outreach --- science journalism --- women in science}.

% \subsection{Texto principal}

% Destacamos algunos puntos del manual de estilo.

% \begin{itemize}
%  \item La primera unidad se separa de la magnitud por un espacio inseparable (\verb|~|). Las unidades subsiguientes van separadas entre si por semi-espacios (\verb|\,|). Las magnitudes deben escribirse en roman (\verb|\mathrm{km}|), estar abreviadas, no contener punto final, y usar potencias negativas para unidades que dividen. Como ejemplo de aplicación de todas estas normas considere: $c \approx 3 \times 10^8~\mathrm{m\,s}^{-1}$ (\verb|$c \approx 3\times 10^8~\mathrm{m\,s}^{-1}$|).
%  \item Para incluir una expresión matemática o ecuación en el texto, sin importar su extensión, se requiere del uso de solo dos signos \verb|$|, uno al comienzo y otro al final. Esto genera el espaciado y tipografía adecuadas para cada detalle de la frase.
%   \item Para separar parte entera de decimal en números utilizar un punto (no coma).
%   \item Para grandes números, separar en miles usando el espacio reducido; ej.: $1\,000\,000$ (\verb+$1\,000\,000$+).
%   \item Las abreviaturas van en mayúsculas; ej.: UV, IR.
%   \item Para abreviar ``versus'' utilizar ``vs.'' y no ``Vs.''.
%   \item Las comillas son dobles y no simples; ej.: ``palabra'', no `palabra'.
%   \item Las llamadas a figuras y tablas comienzan con mayúscula si van seguidas del número correspon\-dien\-te. Si la palabra ``Figura'' está al inicio de una sentencia, se debe escribir completa. En otro caso, se escribe "Fig." (o bien Ec. o Tabla en caso de las ecuaciones y tablas).
%   \item Especies atómicas; ej.: \verb|He {\sc ii}| (He {\sc ii}).
%   \item Nombres de {\sc paquetes} y {\sc rutinas} de {\em software} con tipografía {\em small caps} (\verb|\sc|).
%   \item Nombres de {\sl misiones espaciales} con tipografía {\em slanted} (\verb|\sl|)
% \end{itemize}

% \subsection{Ecuaciones y símbolos matemáticos}

% Las ecuaciones deben enumerarse utilizando el entorno \verb|\begin{equation} ... \end{equation}|, o similares (\verb|{align}, {eqnarray}|, etc.). Las ecuaciones deben llevar al final la puntuación gramatical correspondiente, como parte de la frase que conforman. Como se detalla más arriba, para expresiones matemáticas o ecuaciones insertas en el texto, encerrarlas únicamente entre dos símbolos \verb|$|, utilizando \verb|\mathrm{}| para las unidades. Los vectores deben ir en ``negrita'' utilizando \verb|\mathbf{}|.

% \subsection{Tablas}

% Las tablas no deben sobrepasar los márgenes establecidos para el texto (ver Tabla \ref{tabla1}), y {no se pueden usar modificadores del tamaño de texto}.
% En las tablas se debe incluir cuatro líneas: dos superiores, una inferior y una que separa el encabezado. Se pueden confeccionar tablas de una columna (\verb|\begin{table}|) o de todo el ancho de la página (\verb|\begin{table*}|).

% \begin{table}[!t]
% \centering
% \caption{Ejemplo de tabla. Notar en el archivo fuente el manejo de espacios a fin de lograr que la tabla no exceda el margen de la columna de texto.}
% \begin{tabular}{lccc}
% \hline\hline\noalign{\smallskip}
% \!\!Date & \!\!\!\!Coronal $H_r$ & \!\!\!\!Diff. rot. $H_r$& \!\!\!\!Mag. clouds $H_r$\!\!\!\!\\
% & \!\!\!\!10$^{42}$ Mx$^{2}$& \!\!\!\!10$^{42}$ Mx$^{2}$ & \!\!\!\!10$^{42}$ Mx$^{2}$ \\
% \hline\noalign{\smallskip}
% \!\!07 July  &  -- & (2) & [16,64]\\
% \!\!03 August& [5,11]& 3 & [10,40]\\
% \!\!30 August & [17,23] & 3& [4,16]\\
% \!\!25 September & [9,12] & 1 & [10,40]\\
% \hline
% \end{tabular}
% \label{tabla1}
% \end{table}

% \subsection{Figuras}

% Las figuras deberán prepararse en formatos ``jpg'', ``png'' o ``pdf'', siendo este último el de preferencia. Deben incluir todos los elementos que posibiliten su correcta lectura, tales como escalas y nombres de los ejes coordenados, códigos de líneas, símbolos, etc.  Verifique que la resolución de imagen sea adecuada. El tamaño de letra de los textos de la figura debe ser igual o mayor que en el texto del epígrafe (ver p.ej. la Fig.~\ref{Figura}). Al realizar figuras a color, procure que no se pierda información cuando se visualiza en escala de grises (como en la versión impresa del BAAA). Por ejemplo, en la Fig.~\ref{Figura}, las curvas sólidas podrían diferenciarse con símbolos diferentes (círculo en una y cuadrado en otra), y una de las curvas punteadas podría ser rayada. Para figuras tomadas de otras pu\-bli\-ca\-cio\-nes, envíe a los editores del BAAA el permiso correspondiente y cítela como exige la publicación original 

%%%%%%%%%%%%%%%%%%%%%%%%%%%%%%%%%%%%%%%%%%%%%%%%%%%%%%%%%%%%%%%%%%%%%%%%%%%%%%
% Para figuras de dos columnas use \begin{figure*} ... \end{figure*}         %
%%%%%%%%%%%%%%%%%%%%%%%%%%%%%%%%%%%%%%%%%%%%%%%%%%%%%%%%%%%%%%%%%%%%%%%%%%%%%%

% \begin{figure}[!t]
% \centering
% \includegraphics[width=\columnwidth]{ejemplo_figura_Hough_etal.pdf}
% \caption{El tamaño de letra en el texto y en los valores numéricos de los ejes es similar al tamaño de letra de este epígrafe. Si utiliza más de un panel, explique cada uno de ellos; ej.: \emph{Panel superior:} explicación del panel superior. Figura reproducida con permiso de \cite{Hough_etal_BAAA_2020}.}
% \label{Figura}
% \end{figure}

% \subsection{Referencias cruzadas}\label{ref}

% Su artículo debe emplear referencias cruzadas utilizando la herramienta  {\sc bibtex}. Para ello elabore un archivo (como el ejemplo incluido: {\tt bibliografia.bib}) conteniendo las referencias {\sc bibtex} utilizadas en el texto. Incluya el nombre de este archivo en el comando \LaTeX{} de inclusión de bibliografía (\verb|\bibliography{bibliografia}|). 

% Recuerde que la base de datos ADS contiene las entradas de {\sc bibtex}  para todos los artículos. Se puede acceder a ellas mediante el enlace ``{\em Export Citation}''.

% El estilo de las referencias se aplica automáticamente a través del archivo de estilo incluido (baaa.bst). De esta manera, las referencias generadas tendrán la forma co\-rrec\-ta para un autor \citep{hubble_expansion_1929}, dos autores \citep{penzias_cmb_1965,penzias_cmb_II_1965}, tres autores \citep{navarro_NFW_1997} y muchos autores \citep{riess_SN1a_1998}, \citep{Planck_2016}.

% \begin{acknowledgement}
% Los agradecimientos deben agregarse usando el entorno correspondiente (\texttt{acknowledgement}).
% \end{acknowledgement}

%%%%%%%%%%%%%%%%%%%%%%%%%%%%%%%%%%%%%%%%%%%%%%%%%%%%%%%%%%%%%%%%%%%%%%%%%%%%%%
%  ******************* Bibliografía / Bibliography ************************  %
%                                                                            %
%  -Ver en la sección 3 "Bibliografía" para mas información.                 %
%  -Debe usarse BIBTEX.                                                      %
%  -NO MODIFIQUE las líneas de la bibliografía, salvo el nombre del archivo  %
%   BIBTEX con la lista de citas (sin la extensión .BIB).                    %
%                                                                            %
%  -BIBTEX must be used.                                                     %
%  -Please DO NOT modify the following lines, except the name of the BIBTEX  %
%  file (without the .BIB extension).                                       %
%%%%%%%%%%%%%%%%%%%%%%%%%%%%%%%%%%%%%%%%%%%%%%%%%%%%%%%%%%%%%%%%%%%%%%%%%%%%%% 

\bibliographystyle{baaa}
\small
\bibliography{iglesias}
 
\end{document}
