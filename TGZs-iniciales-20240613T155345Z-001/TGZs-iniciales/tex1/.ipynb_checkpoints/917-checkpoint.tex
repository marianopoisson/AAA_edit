
%%%%%%%%%%%%%%%%%%%%%%%%%%%%%%%%%%%%%%%%%%%%%%%%%%%%%%%%%%%%%%%%%%%%%%%%%%%%%%
%  ************************** AVISO IMPORTANTE **************************    %
%                                                                            %
% Éste es un documento de ayuda para los autores que deseen enviar           %
% trabajos para su consideración en el Boletín de la Asociación Argentina    %
% de Astronomía.                                                             %
%                                                                            %
% Los comentarios en este archivo contienen instrucciones sobre el formato   %
% obligatorio del mismo, que complementan los instructivos web y PDF.        %
% Por favor léalos.                                                          %
%                                                                            %
%  -No borre los comentarios en este archivo.                                %
%  -No puede usarse \newcommand o definiciones personalizadas.               %
%  -SiGMa no acepta artículos con errores de compilación. Antes de enviarlo  %
%   asegúrese que los cuatro pasos de compilación (pdflatex/bibtex/pdflatex/ %
%   pdflatex) no arrojan errores en su terminal. Esta es la causa más        %
%   frecuente de errores de envío. Los mensajes de "warning" en cambio son   %
%   en principio ignorados por SiGMa.                                        %
%                                                                            %
%%%%%%%%%%%%%%%%%%%%%%%%%%%%%%%%%%%%%%%%%%%%%%%%%%%%%%%%%%%%%%%%%%%%%%%%%%%%%%

%%%%%%%%%%%%%%%%%%%%%%%%%%%%%%%%%%%%%%%%%%%%%%%%%%%%%%%%%%%%%%%%%%%%%%%%%%%%%%
%  ************************** IMPORTANT NOTE ******************************  %
%                                                                            %
%  This is a help file for authors who are preparing manuscripts to be       %
%  considered for publication in the Boletín de la Asociación Argentina      %
%  de Astronomía.                                                            %
%                                                                            %
%  The comments in this file give instructions about the manuscripts'        %
%  mandatory format, complementing the instructions distributed in the BAAA  %
%  web and in PDF. Please read them carefully                                %
%                                                                            %
%  -Do not delete the comments in this file.                                 %
%  -Using \newcommand or custom definitions is not allowed.                  %
%  -SiGMa does not accept articles with compilation errors. Before submission%
%   make sure the four compilation steps (pdflatex/bibtex/pdflatex/pdflatex) %
%   do not produce errors in your terminal. This is the most frequent cause  %
%   of submission failure. "Warning" messsages are in principle bypassed     %
%   by SiGMa.                                                                %
%                                                                            % 
%%%%%%%%%%%%%%%%%%%%%%%%%%%%%%%%%%%%%%%%%%%%%%%%%%%%%%%%%%%%%%%%%%%%%%%%%%%%%%

\documentclass[baaa]{baaa}

%%%%%%%%%%%%%%%%%%%%%%%%%%%%%%%%%%%%%%%%%%%%%%%%%%%%%%%%%%%%%%%%%%%%%%%%%%%%%%
%  ******************** Paquetes Latex / Latex Packages *******************  %
%                                                                            %
%  -Por favor NO MODIFIQUE estos comandos.                                   %
%  -Si su editor de texto no codifica en UTF8, modifique el paquete          %
%  'inputenc'.                                                               %
%                                                                            %
%  -Please DO NOT CHANGE these commands.                                     %
%  -If your text editor does not encodes in UTF8, please change the          %
%  'inputec' package                                                         %
%%%%%%%%%%%%%%%%%%%%%%%%%%%%%%%%%%%%%%%%%%%%%%%%%%%%%%%%%%%%%%%%%%%%%%%%%%%%%%
 
\usepackage[pdftex]{hyperref}
\usepackage{subfigure}
\usepackage{natbib}
\usepackage{helvet,soul}
\usepackage[font=small]{caption}

%%%%%%%%%%%%%%%%%%%%%%%%%%%%%%%%%%%%%%%%%%%%%%%%%%%%%%%%%%%%%%%%%%%%%%%%%%%%%%
%  *************************** Idioma / Language **************************  %
%                                                                            %
%  -Ver en la sección 3 "Idioma" para mas información                        %
%  -Seleccione el idioma de su contribución (opción numérica).               %
%  -Todas las partes del documento (titulo, texto, figuras, tablas, etc.)    %
%   DEBEN estar en el mismo idioma.                                          %
%                                                                            %
%  -Select the language of your contribution (numeric option)                %
%  -All parts of the document (title, text, figures, tables, etc.) MUST  be  %
%   in the same language.                                                    %
%                                                                            %
%  0: Castellano / Spanish                                                   %
%  1: Inglés / English                                                       %
%%%%%%%%%%%%%%%%%%%%%%%%%%%%%%%%%%%%%%%%%%%%%%%%%%%%%%%%%%%%%%%%%%%%%%%%%%%%%%

\contriblanguage{0}

%%%%%%%%%%%%%%%%%%%%%%%%%%%%%%%%%%%%%%%%%%%%%%%%%%%%%%%%%%%%%%%%%%%%%%%%%%%%%%
%  *************** Tipo de contribución / Contribution type ***************  %
%                                                                            %
%  -Seleccione el tipo de contribución solicitada (opción numérica).         %
%                                                                            %
%  -Select the requested contribution type (numeric option)                  %
%                                                                            %
%  1: Artículo de investigación / Research article                           %
%  2: Artículo de revisión invitado / Invited review                         %
%  3: Mesa redonda / Round table                                             %
%  4: Artículo invitado  Premio Varsavsky / Invited report Varsavsky Prize   %
%  5: Artículo invitado Premio Sahade / Invited report Sahade Prize          %
%  6: Artículo invitado Premio Sérsic / Invited report Sérsic Prize          %
%%%%%%%%%%%%%%%%%%%%%%%%%%%%%%%%%%%%%%%%%%%%%%%%%%%%%%%%%%%%%%%%%%%%%%%%%%%%%%

\contribtype{1}

%%%%%%%%%%%%%%%%%%%%%%%%%%%%%%%%%%%%%%%%%%%%%%%%%%%%%%%%%%%%%%%%%%%%%%%%%%%%%%
%  ********************* Área temática / Subject area *********************  %
%                                                                            %
%  -Seleccione el área temática de su contribución (opción numérica).        %
%                                                                            %
%  -Select the subject area of your contribution (numeric option)            %
%                                                                            %
%  1 : SH    - Sol y Heliosfera / Sun and Heliosphere                        %
%  2 : SSE   - Sistema Solar y Extrasolares  / Solar and Extrasolar Systems  %
%  3 : AE    - Astrofísica Estelar / Stellar Astrophysics                    %
%  4 : SE    - Sistemas Estelares / Stellar Systems                          %
%  5 : MI    - Medio Interestelar / Interstellar Medium                      %
%  6 : EG    - Estructura Galáctica / Galactic Structure                     %
%  7 : AEC   - Astrofísica Extragaláctica y Cosmología /                      %
%              Extragalactic Astrophysics and Cosmology                      %
%  8 : OCPAE - Objetos Compactos y Procesos de Altas Energías /              %
%              Compact Objetcs and High-Energy Processes                     %
%  9 : ICSA  - Instrumentación y Caracterización de Sitios Astronómicos
%              Instrumentation and Astronomical Site Characterization        %
% 10 : AGE   - Astrometría y Geodesia Espacial
% 11 : ASOC  - Astronomía y Sociedad                                             %
% 12 : O     - Otros
%
%%%%%%%%%%%%%%%%%%%%%%%%%%%%%%%%%%%%%%%%%%%%%%%%%%%%%%%%%%%%%%%%%%%%%%%%%%%%%%

\thematicarea{4}

%%%%%%%%%%%%%%%%%%%%%%%%%%%%%%%%%%%%%%%%%%%%%%%%%%%%%%%%%%%%%%%%%%%%%%%%%%%%%%
%  *************************** Título / Title *****************************  %
%                                                                            %
%  -DEBE estar en minúsculas (salvo la primer letra) y ser conciso.          %
%  -Para dividir un título largo en más líneas, utilizar el corte            %
%   de línea (\\).                                                           %
%                                                                            %
%  -It MUST NOT be capitalized (except for the first letter) and be concise. %
%  -In order to split a long title across two or more lines,                 %
%   please use linebreaks (\\).                                              %
%%%%%%%%%%%%%%%%%%%%%%%%%%%%%%%%%%%%%%%%%%%%%%%%%%%%%%%%%%%%%%%%%%%%%%%%%%%%%%
% Dates
% Only for editors
\received{\ldots}
\accepted{\ldots}




%%%%%%%%%%%%%%%%%%%%%%%%%%%%%%%%%%%%%%%%%%%%%%%%%%%%%%%%%%%%%%%%%%%%%%%%%%%%%%



\title{Rendimiento de métodos semi-automáticos de búsqueda de \\ cúmulos estelares}

%%%%%%%%%%%%%%%%%%%%%%%%%%%%%%%%%%%%%%%%%%%%%%%%%%%%%%%%%%%%%%%%%%%%%%%%%%%%%%
%  ******************* Título encabezado / Running title ******************  %
%                                                                            %
%  -Seleccione un título corto para el encabezado de las páginas pares.      %
%                                                                            %
%  -Select a short title to appear in the header of even pages.              %
%%%%%%%%%%%%%%%%%%%%%%%%%%%%%%%%%%%%%%%%%%%%%%%%%%%%%%%%%%%%%%%%%%%%%%%%%%%%%%

\titlerunning{Rendimiento de métodos automáticos de búsqueda de cúmulos estelares}

%%%%%%%%%%%%%%%%%%%%%%%%%%%%%%%%%%%%%%%%%%%%%%%%%%%%%%%%%%%%%%%%%%%%%%%%%%%%%%
%  ******************* Lista de autores / Authors list ********************  %
%                                                                            %
%  -Ver en la sección 3 "Autores" para mas información                       % 
%  -Los autores DEBEN estar separados por comas, excepto el último que       %
%   se separar con \&.                                                       %
%  -El formato de DEBE ser: S.W. Hawking (iniciales luego apellidos, sin     %
%   comas ni espacios entre las iniciales).                                  %
%                                                                            %
%  -Authors MUST be separated by commas, except the last one that is         %
%   separated using \&.                                                      %
%  -The format MUST be: S.W. Hawking (initials followed by family name,      %
%   avoid commas and blanks between initials).                               %
%%%%%%%%%%%%%%%%%%%%%%%%%%%%%%%%%%%%%%%%%%%%%%%%%%%%%%%%%%%%%%%%%%%%%%%%%%%%%%

\author{
M. Chiarpotti\inst{1,2} \&
A.E. Piatti\inst{1,2}}

\authorrunning{Chiarpotti y Piatti}

%%%%%%%%%%%%%%%%%%%%%%%%%%%%%%%%%%%%%%%%%%%%%%%%%%%%%%%%%%%%%%%%%%%%%%%%%%%%%%
%  **************** E-mail de contacto / Contact e-mail *******************  %
%                                                                            %
%  -Por favor provea UNA ÚNICA dirección de e-mail de contacto.              %
%                                                                            %
%  -Please provide A SINGLE contact e-mail address.                          %
%%%%%%%%%%%%%%%%%%%%%%%%%%%%%%%%%%%%%%%%%%%%%%%%%%%%%%%%%%%%%%%%%%%%%%%%%%%%%%

\contact{mati.charpo@gmail.com}

%%%%%%%%%%%%%%%%%%%%%%%%%%%%%%%%%%%%%%%%%%%%%%%%%%%%%%%%%%%%%%%%%%%%%%%%%%%%%%
%  ********************* Afiliaciones / Affiliations **********************  %
%                                                                            %
%  -La lista de afiliaciones debe seguir el formato especificado en la       %
%   sección 3.4 "Afiliaciones".                                              %
%                                                                            %
%  -The list of affiliations must comply with the format specified in        %          
%   section 3.4 "Afiliaciones".                                              %
%%%%%%%%%%%%%%%%%%%%%%%%%%%%%%%%%%%%%%%%%%%%%%%%%%%%%%%%%%%%%%%%%%%%%%%%%%%%%%

\institute{Instituto Interdisciplinario de Ciencias Básicas, CONICET--UNCuyo, Argentina
\and 
Consejo Nacional de Investigaciones Científicas y Técnicas, Argentina}

%%%%%%%%%%%%%%%%%%%%%%%%%%%%%%%%%%%%%%%%%%%%%%%%%%%%%%%%%%%%%%%%%%%%%%%%%%%%%%
%  *************************** Resumen / Summary **************************  %
%                                                                            %
%  -Ver en la sección 3 "Resumen" para mas información                       %
%  -Debe estar escrito en castellano y en inglés.                            %
%  -Debe consistir de un solo párrafo con un máximo de 1500 (mil quinientos) %
%   caracteres, incluyendo espacios.                                         %
%                                                                            %
%  -Must be written in Spanish and in English.                               %
%  -Must consist of a single paragraph with a maximum  of 1500 (one thousand %
%   five hundred) characters, including spaces.                              %
%%%%%%%%%%%%%%%%%%%%%%%%%%%%%%%%%%%%%%%%%%%%%%%%%%%%%%%%%%%%%%%%%%%%%%%%%%%%%%

\resumen{En este trabajo evaluamos el rendimiento de métodos computacionales de detección de cúmulos estelares utilizando la base de datos infrarroja VVV. La evaluación de dicha capacidad de detección la llevamos a cabo comparando los resultados obtenidos con métodos computacionales que desarrollamos con aquellos basados en la inspección visual de las imágenes VVV. De la comparación entre los resultados obtenidos a partir del método computacional desarrollado y los resultados que provienen de la detección de candidatos a cúmulo estelar por inspección visual, encontramos que el método computacional confirmó un 60\% de los candidatos seleccionados. Estos candidatos resultaron ser sobredensidades estelares que destacan, por lo menos, cinco veces  por sobre la densidad media del fondo del cielo, lo cual interpretamos como un umbral de sensibilidad del ojo humano.}

\abstract{In this work we evaluate the performance of computational methods for detecting star clusters using the VVV infrared database. We carried out the assessment of detecting star cluster candidates by comparing the results obtained from computational methods and those from visual inspection of the VVV images. We found that the computational methods recovered 60\% of visual detected candidates. These candidates turned out to be stellar overdensities that stand out, at least, five times above the average density of the background of the sky, which we interpret as a threshold for the sensitivity of the human eye. }

%\abstract{This is the author guide to prepare articles for the \textit{Boletín de la Asociación Argentina de Astronomía} (BAAA), intended also as a macro for Volume 65. Please read carefully its content to prevent the most frequent style errors. Please also read carefully the comments preceded by the symbol ``\%"in the \LaTeX{} source file of this document. This is the only format accepted for article submission, that must be done via the BAAA Manuscript Management System (SiGMa).}

%%%%%%%%%%%%%%%%%%%%%%%%%%%%%%%%%%%%%%%%%%%%%%%%%%%%%%%%%%%%%%%%%%%%%%%%%%%%%%
%                                                                            %
%  Seleccione las palabras clave que describen su contribución. Las mismas   %
%  son obligatorias, y deben tomarse de la lista de la American Astronomical %
%  Society (AAS), que se encuentra en la página web indicada abajo.          %
%                                                                            %
%  Select the keywords that describe your contribution. They are mandatory,  %
%  and must be taken from the list of the American Astronomical Society      %
%  (AAS), which is available at the webpage quoted below.                    %
%                                                                            %
%  https://journals.aas.org/keywords-2013/                                   %
%                                                                            %
%%%%%%%%%%%%%%%%%%%%%%%%%%%%%%%%%%%%%%%%%%%%%%%%%%%%%%%%%%%%%%%%%%%%%%%%%%%%%%

\keywords{open clusters and associations: general ---  stars: fundamental parameters ---  techniques: photometric}

\begin{document}

\maketitle
\section{Introducción}
\par
El estudio de cúmulos estelares ha sido fundamental desde tiempos inmemoriales para mejorar nuestro conocimiento acerca de la formación, estructura, evolución dinámica y química de las galaxias. Particularmente, el análisis de los cúmulos abiertos  es importante para conocer el pasado y presente de las propiedades de la Vía Láctea. Para la detección y posterior análisis de nuevos candidatos a cúmulos estelares a partir de relevamientos en el rango visible del espectro electromagnético, se utilizan actualmente programas de búsquedas automáticos o semiautomáticos con resultados satisfactorios (\citealt{ferreira2020discovery}). Sin embargo, estos métodos no han sido aún evaluados extensamente para detectar nuevos candidatos a cúmulos estelares en la región infrarroja del espectro. La región infrarroja es favorable para estudiar zonas de la Vía Láctea marcadamente afectadas por extinción interestelar, en la cual también abundan un gran número de estrellas, como por ejemplo las regiones centrales de la Galaxia. Estas características imponen nuevos desafíos a la búsqueda de candidatos a cúmulos estelares. 

%\begin{figure}[!t]
%\centering

%\includegraphics[width=\textwidth]{Areasurvey.png}
%\caption{Distribución espacial de los datos VVV. Cada cuadro rojo representa un  \emph{tile}.  En negro se representan los candidatos a cúmulos detectados por Borissova et al. (2011) en su trabajo. Fuente: Borissova et al. (2011).}
%\label{Figura}

%\end{figure}
%Esto se debe a que los mismos se han formado a lo largo de toda vida de la Vía Láctea, como así también a lo largo de su extensión, compartiendo enriquecimiento químico y evolución dinámica en el entorno de formación.
\par
\cite{borissova2011new} identificaron, con un método de detección visual, 96 candidatos a cúmulos estelares y grupos estelares en el infrarrojo, utilizando los datos recolectados por \textit{Visible and Infrared Survey Telescope for Astronomy, Variables in the Vía Láctea} (VVV:Fig. 1). El VVV es un relevamiento público  que tiene como objetivo observar homogéneamente la región central y el disco interno de la Vía Láctea, utilizando el telescopio VISTA de 4m ubicado en el Observatorio Europeo Austral (ESO, Chile). Cada región aproximadamente cuadrada (2x2 grados) cubierta por el mosaico de detectores empleados por el telescopio VISTA se denomina \textit{tile}. 
Los datos que proporciona para cientos de millones de estrellas observadas en esta región de la galaxia consisten en la posición, movimientos propios y magnitudes en 5 filtros diferentes (Z, Y, J, H, Ks).  \cite{borissova2011new}
utilizaron solamente estrellas con Ks \textless 13.5 mag.
\par
%Por otro lado, en el rango óptico del espectro los métodos de detección están muy desarrollados, tanto los manuales (interactivos) como los automáticos y semiautomáticos. En otras palabras, podemos decir que los métodos automáticos o semiautomáticos en el rango óptico del espectro han sido extensamente utilizados y son fiables a la hora de detectar un candidato a cúmulo estelar. Sin embargo, estos métodos no han sido aún utilizados para detectar nuevos candidatos a cúmulos estelares en la región infrarroja del espectro de donde son los CA que que utilizaremos para hacer nuestra comparación. 
\par

%El VVV tiene como objetivo escanear las zonas mas pobladas y difíciles de estudiar de la Vía Láctea como son el bulbo y el disco galáctico, en donde además se supone que la actividad de formación estelar es alta. Una gran parte de las estrellas observadas con VVV han sido detectadas por primera vez; la radiación que proviene de ellas en el rango óptico del espectro es absorbida total o parcialmente por el polvo interestelar que se interpone entre ellas y nosotros, y por este motivo no son observables en el rango óptico del espectro electromagnético.
\par 
% Este proyecto observacional fue utilizado permiten avances en el desarrollo de nuevos métodos de detección de cúmulos estelares. Tradicionalmente las observaciones fotométricas realizadas con telescopios eran solamente empleando filtros en el rango óptico del espectro, lo cual permitía observar apenas un porcentaje pequeño de las estrellas existentes. En este rango del espectro electromagnético, los métodos de detección están muy desarrollados, tanto los manuales (interactivos) como los automáticos y semiautomáticos. En otras palabras, podemos decir que los métodos automáticos o semiautomáticos en el rango óptico del espectro han sido extensamente utilizados y son fiables a la hora de detectar un candidato a cúmulo estelar. Sin embargo, estos métodos no han sido aún utilizados para detectar nuevos candidatos a cúmulos estelares en la región infrarroja del espectro. Debido a que la apariencia (sobredensidad estelar) de un cúmulo estelar en el visible y en el infrarrojo  son diferentes, es posible que cúmulos detectados en el visible no lo sean en el infrarrojo y viceversa. Esto se debe a que los cúmulos estelares ubicados en dirección al centro de la Vía Láctea están mas afectados por la absorción interestelar o por nubes de gas estelar remanentes de los mismos. 
 
 
 \begin{figure*}[!t]
\centering

\includegraphics[width=0.8\textwidth]{VVVSURVEY.png}
\caption{Esquema de las regiones de la Galaxia observada por VVV representadas con grillas rojas. Dichas regiones fueron relevadas uniformemente con campos de 2 grados de lado aproximadamente llamados \textit{tiles}. En amarillo se representan los \textit{tiles} seleccionados en este trabajo (d070, d105, d088 en orden de izquierda a derecha). Imagen obtenida de la página oficial de \textit{VVV survey} (https://vvvsurvey.org/about/survey-area/).}
\label{Figura}

\end{figure*}
 \par
\cite{piatti2016vmc} desarrollaron un método de búsqueda de cúmulos semiautomático con el cual identificaron 38 candidatos a cúmulos estelares en el infrarrojo utilizando los datos proporcionados por VISTA para el relevamiento \textit{Vista Magellanic Clouds} (VMC) en una zona altamente enrojecida de la Nube Mayor de Magallanes. Para su desarrollo, utilizaron un programa de búsqueda de sobredensidades basado en una librería de \textit{Machinelearning} llamado {\sc Kernel Density Estimator (KDE), el} cual utiliza como variables principales el ancho de banda o \textit{bandwidth, h,} y la función de distribución de densidad estelar (gaussiana). Para obtener los valores de $h$ apropiados confeccionaron inicialmente una lista de 68 cúmulos estelares conocidos en la región estudiada, utilizando el catálogo de \cite{bica2008general}, de los cuales extrajeron los valores de los radios y densidades medias. En particular, utilizaron 3 valores de $h$ diferentes, correspondientes a los valores mínimo, intermedio y máximo de los radios de los cúmulos previamente conocidos. De todas las sobredensidades detectadas por {\sc KDE} en las imágenes VMC, \cite{piatti2016vmc} seleccionaron aquellas cuya densidad y extensión (radio) se encontraran dentro de los rangos respectivos de los cúmulos estelares conocidos en la región analizada. De la búsqueda de nuevos candidatos en la misma región del cielo, pero a partir de datos en el infrarrojo, \cite{piatti2016vmc} detectaron 143 sobredensidades estelares nuevas. Posteriormente, realizaron una limpieza de posibles estrellas de campo, obteniendo así 38 objetos candidatos a cúmulo estelar genuinos.
\par
En el presente trabajo desarrollaremos un método de detección y análisis automático de cúmulos estelares y lo aplicamos a los mismos datos utilizados por
\cite{borissova2011new}. El método propuesto se basará principalmente en la implementación de {\sc KDE}. A partir de su utilización y de sus resultados comparados con aquéllos de \cite{borissova2011new} evaluamos su rendimiento en la detección de cúmulos estelares en la región infrarroja del espectro electromagnético.


%La 65a reunión anual de la Asociación Argentina de Astronomía (AAA) se desarrolló del 18 al 22 de septiembre de 2023 en la ciudad de San Juan. Durante la misma, se expusieron 68 trabajos en forma de presentaciones orales y 133  trabajos en forma de presentaciones murales, incluyendo 14 charlas invitadas (dos correspondientes a los Premios Sahade y Sérsic). Invitamos cordialmente a los expositores de dicha reunión a remitir sus contribuciones en forma escrita, para que puedan ser consideradas para su publicación en el {Vol. 65 del BAAA.}

%El Comité Editorial de este volumen está integrado por Cristina H. Mandrini como Editora en Jefe. Claudia E. Boeris como Secretaria Editorial y Mariano Poisson como Técnico Editorial. A éstos se incorporan los Editores Asociados: Andrea P. Buccino, Gabriela Castelletti, Sofía A. Cora, Héctor J. Martínez y Mariela C. Vieytes. El Editor Invitado es Hernán Muriel, quien se desempeñó como Presidente del Comité Organizador Científico de la reunión. 

%Al considerar el envío de su contribución, tener en cuenta los siguientes puntos:
%\begin{itemize}
 %   \item La carga de contribuciones y su seguimiento durante la etapa de revisión, se realiza exclusivamente utilizando el Sistema de Gestión de Manuscritos de la AAA  (SiGMa)\footnote{\url{http://sigma.fcaglp.unlp.edu.ar/}}. 
  %  \item Las contribuciones serán revisadas por árbitros externos asignados por los editores (excepto las correspondientes a informes invitados, premios y mesas redondas). Los árbitros constatarán, entre otros aspectos, la originalidad de su contribución. No se aceptarán contribuciones ya publicadas o enviadas a publicar a otra revista. 
   % \item Los manuscritos aceptados formarán parte de la base de publicaciones ADS.    
   % \item  El BAAA está regulado por el Reglamento de Publicaciones\footnote{\url{http://astronomiaargentina.org.ar/uploads/docs/reglamento_publicaciones.pdf}} de la AAA, artículos 2 al 6.
%\end{itemize}
                                                                   
%Agradecemos desde ya el envío de contribuciones en tiempo y forma, ayudando a lograr que la próxima edición de la única publicación de astronomía profesional de la Argentina se publique lo antes posible.


\section{Método}
La detección de candidatos a cúmulos estelares la llevamos a cabo utilizando como base de datos los \textit{tiles} d070, d088, d105 del relevamiento VISTA VVV, mostrados con amarillo en la Fig 1. Elegimos estos \textit{tiles} porque contienen un número significativo de candidatos a cúmulos estelares identificados visualmente por \cite{borissova2011new}, lo cual permite evaluar mejor estadísticamente los métodos semiautomáticos de búsqueda de cúmulos. Cada \textit{tile} contiene 5 candidatos detectados visualmente por \cite{borissova2011new}. \par
El análisis de los 15 candidatos a cúmulos estelares lo llevamos a cabo con un programa de detección semiautomático desarrollado ad-hoc, que tiene en cuenta sobredensidades estelares mayores a un cierto umbral, y la correspondiente distribución de las estrellas que componen una sobredensidad en el diagrama color-magnitud. Luego, comparamos nuestros resultados con los de \cite{borissova2011new}. \par
A continuación describimos la metodología, paso a paso.

\subsection{Comprobación de candidatos} 
\par
En primer lugar seleccionamos una región de  $0.11$ grados de radio alrededor de las coordenadas centrales de los 15 candidatos a cúmulos estelares seleccionados. Dichas coordenadas centrales las tomamos de \cite{borissova2011new}. El tamaño del campo lo elegimos de modo que en ellos se incluyera no solamente el campo del candidato a cúmulo estelar, sino también una buena parte de su campo circundante, por razones que resultarán claras más adelante. 
Luego, graficamos las posiciones (R.A. y Dec., por sus siglas en inglés) de todas las estrellas observadas en el campo de cada candidato a cúmulo estelar en un área cuadrada de $0.11$ grados de lado. Sobre estos campos seleccionados aplicamos {\sc KDE} para reconocer dichas sobredensidades estelares. Además, realizamos un conteo estelar para cada candidato para verificar los datos extraídos de \cite{borissova2011new}. \par

\subsection{Mapas de densidad estelar}
{\sc KDE} provee de información muy valiosa para detectar la existencia de sobredensidades reales de entre cientos de miles de estrellas examinadas en la base de datos VVV. Como resultado de diferentes corridas de \textit{{\sc KDE}}, obtuvimos las posiciones centrales, en ascensión recta y declinación, (R.A. y Dec.) de las sobredensidades estelares detectadas. Para filtrar sobredensidades estelares espúreas, por ejemplo, aquellas debidas a fluctuaciones de la densidad estelar del campo, sustrajimos a las densidades obtenidas por \textit{{\sc KDE}} el valor medio de la densidad real de cada campo en los alrededores del candidato a cúmulo estelar, y dividimos dicho valor por la dispersión  de la densidad de las estrellas del campo, que representamos con la variable sigma ($\sigma$). La Fig. 2 muestra la distribución de densidad "$\sigma$'' en función de la posición espacial de las estrellas.

\begin{figure}
\centering
\includegraphics[width=\columnwidth]{CL35.png}
\caption{Mapa de densidad estelar generado por {\sc KDE} para el candidato a cúmulo estelar Cl35. El círculo negro está centrado en el objeto y su radio es el estimado por \cite{borissova2011new}. El Norte está hacia arriba y el Este hacia la izquierda.}
\label{fig:N_publications}
\end{figure}


El modo concreto deutilizar \textit{{\sc KDE}} lo detallamos a continuación: Inicialmente generamos una grilla de celdas cuadradas distribuidas en todo el campo a explorar; en cada una de ellas \textit{{\sc KDE}} aplica una función de distribución de densidades definida previamente. El tamaño de las celdas cuadradas elegido fue ($\Delta Dec.  , \Delta R.A.) = (826{\rm pixel},826{\rm pixel})$ (0.479 segundos de lado), y el mismo resultó suficientemente grande como para evitar detectar fluctuaciones espúreas y suficientemente chico como para detectar sobredensidades reales pequeñas. De entre todos los parámetros que tuvimos que variar dentro de {\sc KDE}, éste resultó el más sensible en cuanto al tiempos de ejecución que demandó cada corrida de \textit{{\sc KDE}}.
\par 
Para utilizar un valor de ancho de banda apropiado decidimos adoptar un valor igual a 4 veces el radio más pequeño de los candidatos a cúmulo estelar, expresado en grados, de modo que cada sobredensidad estelar asociada a un candidato a cúmulo estelar pueda ser mapeada por \textit{{\sc KDE}} por una cantidad razonable (número de celdas) de funciones de distribución. El candidato con menor radio en nuestra muestra es Cl88 con un valor de $0.0033$grados. En base a esto, adoptamos $h$ = $0.0012$grados.

\subsection{Comparación con diagramas color-magnitud}
Como última condición necesaria para corroborar si efectivamente las sobredensidades estelares reportadas en \cite{borissova2011new} son candidatos a cúmulos estelares, analizamos sus diagramas color-magnitud. Para construir este diagrama, utilizamos las magnitudes en los filtros Ks y J.
\par 
Para confeccionar el diagrama color-magnitud de cada candidato a cúmulo estelar tomamos las estrellas contenidas en un círculo con centro correspondiente a las coordenadas R.A. y Dec. y el radio reportado en \cite{borissova2011new}. El diagrama color-magnitud resultante lo comparamos con aquel de un campo estelar adyacente, para un círculo de igual área. Si el objeto de interés es un candidato a cúmulo estelar, su diagrama color-magnitud contiene secuencias de estrellas que se distinguen del diagrama color magnitud de las estrellas del campo adyacentes al objeto.


\section{Análisis y resultados}
Luego de realizadas las corridas de los métodos semi-automáticos mencionados obtuvimos los siguientes resultados. Las Figs. 3 y 4 muestran los resultados obtenidos a partir de la aplicación del método semi-automático para Cl36. Se realizó un análisis de completitud de {\sc KDE} 
 de forma individual en los 3 \textit{tiles} seleccionados, resultando 100\%, 100\%, y 75\%, respectivamente.

\subsection{Sobredensidades detectadas con {\sc KDE}}
A partir  del conteo estelar observamos que el número de estrellas contabilizadas computacionalmente dentro del radio de cada candidato a cúmulo estelar es aproximadamente un 100 \% mayor que el número total de estrellas miembros estimado por \cite{borissova2011new}. Esta diferencia puede ser debida a las limitaciones del método visual para reconocer estrellas detrás de otras, aunque ellos no mencionan si el número de estrellas miembros obtenido incluye la substracción estadística de estrellas del campo. \par
Los mapas de densidad estelar para los candidatos a cúmulo estelar Cl35, Cl36, Cl37, Cl73, Cl74, Cl86, Cl87, Cl88 muestran picos de sobredensidad centrados en las coordenadas de los candidatos identificados por \cite{borissova2011new}, o poseen múltiples picos dentro del radio respectivo. En cambio, los candidatos Cl33 y Cl34 
presentan sobredensidades con picos que están desplazados ligeramente de las posiciones centrales de los mismos, aunque en todos los casos dentro del radio de cada candidato a cúmulo estelar.  Por otro lado, Cl76 y Cl90 no poseen ningún pico de densidad dentro del radio respectivo, por lo que podemos concluir que estos  candidatos a cúmulo estelar no presentan ninguna sobredensidad.


\subsection{Diagramas color-magnitud}
Realizamos un análisis de los diagramas color-magnitud construidos y concluimos que los candidatos a cúmulo estelar  Cl36 ,Cl37, Cl74, Cl87 y Cl88 poseen secuencias de estrellas que insinúan la presencia de un cúmulo estelar. Mientras que para los candidatos Cl33, Cl34, Cl35, Cl73, Cl74, Cl76, Cl88 y Cl90  no detectamos secuencias de estrellas que puedan pertenecer a un cúmulo estelar. 
\par


\begin{figure*} 
 \centering
  {

    \includegraphics[width=0.45\textwidth]{Cl36Desindad.png}
    \includegraphics[width=0.4\textwidth]{Cl36Distribucion.png}}
 \caption{\emph{Panel izquierdo:} Diagrama de distribución de sobredensidades, \emph{Panel derecho:} Diagrama de distribución espacial de las estrellas pertenecientes al candidato a cúmulo Cl36 y estrellas del campo cercano. El círculo negro representa el radio reportado por \cite{borissova2011new}.}
\end{figure*}


\begin{figure}
\centering
\includegraphics[width=0.9\columnwidth]{Cl36CMD.png}
\caption{Diagrama color-magnitud, sin descontaminar, del candidato a cúmulo estelar Cl36.}
\label{Figura}
\end{figure}
\section{Discusión y conclusiones}
Luego de inspeccionados y analizados los gráficos de distribución espacial y distribución de densidades decidimos descartar los candidatos a cúmulo estelar Cl76 y Cl90, debido a que ninguno cumple con la condición de conformar una sobredensidad estelar; no fueron detectados por {\sc KDE} por lo que no los tuvimos en cuenta en adelante. También descartamos los candidatos a cúmulo estelar Cl77, Cl78 y Cl91 debido a que las estrellas de la base de datos VVV que componen dichos candidatos no está completa y presenta errores. Por otro lado, los candidatos a cúmulo estelar Cl34, Cl35, Cl73, y Cl88 no presentan una secuencia de estrellas propias de un cúmulo estelar en su diagrama color-magnitud, y por lo tanto también los descartamos como candidatos a cúmulo. Finalmente, los 6 candidatos a cúmulo estelar Cl33, Cl36, Cl37, Cl74, Cl87 y Cl66 presentan sobredensidades estelares y diagramas color-magnitud compatibles con los de cúmulos estelares, y por lo tanto confirmamos los mismos como candidatos a cúmulo estelar. Concluimos que de 10 candidatos a cúmulos estelares detectados en \cite{borissova2011new}, solamente 6 fueron detectados por {\sc KDE}. Esto implica que el rendimiento o capacidad de {\sc KDE} para detectar los candidatos a cúmulo estelar de \cite{borissova2011new} es de: $6/10$,  lo cual representa un $60\%$. Este porcentaje hace referencia a  candidatos a cúmulo estelar donde {\sc KDE} detecta sobredensidades bien definidas y sus respectivos diagramas color-magnitud también muestran secuencias compatibles con las de cúmulos estelares. De los candidatos descartados, donde la sobredensidad no es clara o los diagramas color-magnitud de sobredensidades no muestran secuencias claras de un cúmulo estelar no podemos concluir definitivamente que no sean cúmulos estelares. Especulamos que esos candidatos a cúmulo estelar reconocidos por inspección visual pueden estar constituidos por pocas estrellas o embebidos en nebulosas, las cuales fueron identificadas por \cite{borissova2011new} como candidatos a cúmulo estelar.
\par




%Los resultados obtenidos no sólo hacen referencia a la detección de sobredensidades estelares con KDE, sino a la confirmación de que dichas sobredensidades puedan corresponder a un candidato a cúmulo estelar. En 4 de los objetos analizados, KDE detectó efectivamente una sobredensidad, pero las descartamos en base a que sus diagramas color-magnitud no muestran las secuencias esperadas. En este sentido, esas sobredensidades podrían ser fluctuaciones del campo y no constituir genuinamente un cúmulo. De confirmarse con estudios posteriores, se podría concluir que los candidatos identificados por Borissova et al. (2011) no constituyen necesariamente cúmulos estelares genuinos. Nuestro análisis con KDE y diagramas color-magnitud ayudaría a limpiar la lista de candidatos a cúmulo estelar sugerido por Borissova et al. (2011). Ciertamente, esto implica que la identificación visual que ellos realizaron podría contener objetos que aparentemente parecen cúmulos estelares pero no lo son.

\par 
Una de las ventajas que hemos identificado en los métodos computacionales es que las coordenadas centrales de las sobredensidades son estimadas con mayor certeza, al igual que la estimación de sus extensiones (radios). Por ejemplo, las sobredensidades detectadas para los candidatos a cúmulo estelar Cl33 y Cl34 están desplazadas ligeramente de las coordenadas provistas por \cite{borissova2011new}. Por otro lado, hemos detectado que los candidatos a cúmulo estelar estudiados y finalmente confirmados se distinguen de la densidad del fondo del cielo en $\sigma$ $>$ 5. De aquí concluimos que la sensibilidad del ojo humano permitiría distinguir sobredensidades si ellas sobresalen claramente de la distribución de las estrellas del campo.
\par

%El BAAA admite dos categorías de contribución:
%\begin{itemize}
%    \item Breve (4 páginas), correspondiente a comunicación oral o mural.
%    \item Extensa (8 páginas), correspondiente a informe invitado, mesa redonda o premio.
%\end{itemize}
%El límite de páginas especificado para cada categoría aplica aún después de introducir correcciones arbitrales y editoriales. Queda a cargo de los autores hacer los ajustes de extensión que resulten necesarios. {No está permitido el uso de comandos que mo\-di\-fiquen las propiedades de espaciado y tamaño del texto}, tales como $\backslash${\tt small}, $\backslash${\tt scriptsize}, $\backslash${\tt vskip}, etc. 


%Tenga en cuenta los siguientes puntos para la correcta preparación de su manuscrito:
%\begin{itemize}
%    \item Utilice exclusivamente este macro ({\tt articulo-baaa65.tex}), no el de ediciones anteriores. El mismo puede ser descargado desde SiGMa o desde el sistema de edición en línea \href{https://www.overleaf.com/}{\emph{Overleaf}}, como la plantilla titulada  \emph{Boletín Asociación Argentina de Astronomía}.
%   \item Elaborar el archivo fuente (*.tex) de su contribución respetando el %formato especificado en la Sec.~\ref{sec:guia}      
%   \item No está permitido el uso de definiciones o comandos personalizados en \LaTeX{}.
%\end{itemize}

%\subsection{Plazos de recepción de manuscritos}

%La recepción de trabajos correspondientes a comunicación oral o mural se extiende hasta el día {\bf 9 de febrero de 2024} inclusive. Las contribuciones tipo informe invitado, mesa redonda o premio, se recibirán hasta el {\bf 16 de febrero de 2024} inclusive. La recepción finalizará automáticamente en las fechas indicadas, por lo que no se admitirán contribuciones enviadas con fechas posteriores.


%\section{Guía de estilo para el BAAA}\label{sec:guia}

%Al elaborar su manuscrito, siga rigurosamente el estilo definido en esta sección. Esta lista no es exhaustiva, el manual de estilo completo está disponible en la sección \href{http://sigma.fcaglp.unlp.edu.ar/docs/SGM_docs_v01/Surf/index.html}{Instructivos} del SiGMa. Si algún caso no está incluido en el manual de estilo del BAAA, se solicita seguir el estilo de la revista Astronomy \& Astrophysics\footnote{\url{{https://www.aanda.org/for-authors/latex-issues/typography}}}.

%\subsection{Idioma del texto, resumen y figuras}

%El artículo puede escribirse en español o inglés a decisión del autor. El resumen debe escribirse siempre en ambos idiomas. Todas las partes del documento (título, texto, figuras, tablas, etc.)  deben estar en el idioma del texto principal. Al utilizar palabras de un lenguaje diferente al del texto (solo si es inevitable) incluirlas en {\em cursiva}.

%\subsection{Título}

%Inicie en letra mayúscula solo la primera palabra, nombres propios o acrónimos. Procure ser breve, de ser necesario divida el título en múltiples líneas, puede utilizar el corte de línea (\verb|\\|). No agregue punto final al título.

%\subsection{Autores}

%Los autores deben estar separados por comas, excepto el último que se separa con ``\verb|\&|''. El formato es: S.W. Hawking (iniciales luego apellidos, sin comas ni espacios entre las iniciales). Si envía varios artículos, por favor revisar que el nombre aparezca igual en todos ellos, especialmente en apellidos dobles y con guiones.

%\subsection{Afiliaciones}

%El archivo ({\sc ASCII}) {\tt BAAA\_afiliaciones.txt} incluido en este paquete, lista todas las afiliaciones de los autores de esta edición en el formato adoptado por el BAAA. En caso de no encontrar su institución, respete el formato: Instituto (Observatorio o Facultad), Dependencia institucional (para instituciones en Argentina sólo indique las siglas), País (en español).  No incluya punto final en las afiliaciones, excepto si es parte del nombre del país, como por ejemplo: ``EE.UU.".

%\subsection{Resumen}

%Debe consistir de un solo párrafo con un máximo de 1\,500 (mil quinientos) caracteres, incluyendo espacios. Debe estar escrito en castellano y en inglés. No están permitidas las referencias bibliográficas o imágenes. Evite el uso de acrónimos en el resumen. 

%\subsection{Palabras clave: \textit{Keywords}}

%Las palabras clave deben ser escritas en inglés y seleccionarse exclusivamente de la lista de la American Astronomical Society (AAS) \footnote{\url{https://journals.aas.org/keywords-2013/}}. Toda parte indicada entre paréntesis no debe incluirse. Por ejemplo, ``(stars:) binaries (including multiple): close'' debe darse como ``binaries: close''. Palabras que incluyen nombres individuales de objetos lo hacen entre paréntesis, como por ejemplo: ``galaxies: individual (M31)". Respete el uso de letras minúsculas y mayúsculas en el listado de la AAS. Note que el delimitador entre palabras clave es el triple guión. Las {\em keywords} de este artículo ejemplifican todos estos detalles. 

%Finalmente, además de las palabras clave listadas por la AAS, el BAAA incorpora a partir del Vol. 61B las siguientes opciones: {citizen science --- education --- outreach --- science journalism --- women in science}.

%\subsection{Texto principal}

%Destacamos algunos puntos del manual de estilo.

%\begin{itemize}
 %\item La primera unidad se separa de la magnitud por un espacio inseparable (\verb|~|). Las unidades subsiguientes van separadas entre si por semi-espacios (\verb|\,|). Las magnitudes deben escribirse en roman (\verb|\mathrm{km}|), estar abreviadas, no contener punto final, y usar potencias negativas para unidades que dividen. Como ejemplo de aplicación de todas estas normas considere: $c \approx 3 \times 10^8~\mathrm{m\,s}^{-1}$ (\verb|$c \approx 3\times 10^8~\mathrm{m\,s}^{-1}$|).
 %\item Para incluir una expresión matemática o ecuación en el texto, sin importar su extensión, se requiere del uso de solo dos signos \verb|$|, uno al comienzo y otro al final. Esto genera el espaciado y tipografía adecuadas para cada detalle de la frase.
  %\item Para separar parte entera de decimal en números utilizar un punto (no coma).
  %\item Para grandes números, separar en miles usando el espacio reducido; ej.: $1\,000\,000$ (\verb+$1\,000\,000$+).
  %\item Las abreviaturas van en mayúsculas; ej.: UV, IR.
  %\item Para abreviar ``versus'' utilizar ``vs.'' y no ``Vs.''.
  %\item Las comillas son dobles y no simples; ej.: ``palabra'', no `palabra'.
  %\item Las llamadas a figuras y tablas comienzan con mayúscula si van seguidas del número correspon\-dien\-te. Si la palabra ``Figura'' está al inicio de una sentencia, se debe escribir completa. En otro caso, se escribe "Fig." (o bien Ec. o Tabla en caso de las ecuaciones y tablas).
  %\item Especies atómicas; ej.: \verb|He {\sc ii}| (He {\sc ii}).
  %\item Nombres de {\sc paquetes} y {\sc rutinas} de {\em software} con tipografía {\em small caps} (\verb|\sc|).
  %\item Nombres de {\sl misiones espaciales} con tipografía {\em slanted} (\verb|\sl|)
%\end{itemize}

%\subsection{Ecuaciones y símbolos matemáticos}

%Las ecuaciones deben enumerarse utilizando el entorno \verb|\begin{equation} ... \end{equation}|, o similares (\verb|{align}, {eqnarray}|, etc.). Las ecuaciones deben llevar al final la puntuación gramatical correspondiente, como parte de la frase que conforman. Como se detalla más arriba, para expresiones matemáticas o ecuaciones insertas en el texto, encerrarlas únicamente entre dos símbolos \verb|$|, utilizando \verb|\mathrm{}| para las unidades. Los vectores deben ir en ``negrita'' utilizando \verb|\mathbf{}|.

%\subsection{Tablas}


%Las tablas no deben sobrepasar los márgenes establecidos para el texto (ver Tabla \ref{tabla1}), y {no se pueden usar modificadores del tamaño de texto}.
%En las tablas se debe incluir cuatro líneas: dos superiores, una inferior y una que separa el encabezado. Se pueden confeccionar tablas de una columna (\verb|\begin{table}|) o de todo el ancho de la página (\verb|\begin{table*}|).

%\begin{table}[!t]
%\centering
%\caption{Ejemplo de tabla. Notar en el archivo fuente el manejo de espacios a fin de lograr que la tabla no exceda el margen de la columna de texto.}
%\begin{tabular}{lccc}
%\hline\hline\noalign{\smallskip}
%\!\!Date & \!\!\!\!Coronal $H_r$ & \!\!\!\!Diff. rot. $H_r$& \!\!\!\!Mag. clouds $H_r$\!\!\!\!\\
%& \!\!\!\!10$^{42}$ Mx$^{2}$& \!\!\!\!10$^{42}$ Mx$^{2}$ & \!\!\!\!10$^{42}$ Mx$^{2}$ \\
%\hline\noalign{\smallskip}
%\!\!07 July  &  -- & (2) & [16,64]\\
%\!\!03 August& [5,11]& 3 & [10,40]\\
%\!\!30 August & [17,23] & 3& [4,16]\\
%\!\!25 September & [9,12] & 1 & [10,40]\\
%\hline
%\end{tabular}
%\label{tabla1}
%\end{table}

%\subsection{Figuras}

%Las figuras deberán prepararse en formatos ``jpg'', ``png'' o ``pdf'', siendo este último el de preferencia. Deben incluir todos los elementos que posibiliten su correcta lectura, tales como escalas y nombres de los ejes coordenados, códigos de líneas, símbolos, etc.  Verifique que la resolución de imagen sea adecuada. El tamaño de letra de los textos de la figura debe ser igual o mayor que en el texto del epígrafe (ver p.ej. la Fig.~\ref{Figura}). Al realizar figuras a color, procure que no se pierda información cuando se visualiza en escala de grises (como en la versión impresa del BAAA). Por ejemplo, en la Fig.~\ref{Figura}, las curvas sólidas podrían diferenciarse con símbolos diferentes (círculo en una y cuadrado en otra), y una de las curvas punteadas podría ser rayada. Para figuras tomadas de otras pu\-bli\-ca\-cio\-nes, envíe a los editores del BAAA el permiso correspondiente y cítela como exige la publicación original 

%%%%%%%%%%%%%%%%%%%%%%%%%%%%%%%%%%%%%%%%%%%%%%%%%%%%%%%%%%%%%%%%%%%%%%%%%%%%%%
% Para figuras de dos columnas use \begin{figure*} ... \end{figure*}         %
%%%%%%%%%%%%%%%%%%%%%%%%%%%%%%%%%%%%%%%%%%%%%%%%%%%%%%%%%%%%%%%%%%%%%%%%%%%%%%

%\begin{figure}[!t]
%\centering
%\includegraphics[width=\columnwidth]{ejemplo_figura_Hough_etal.pdf}
%\caption{El tamaño de letra en el texto y en los valores numéricos de los ejes es similar al tamaño de letra de este epígrafe. Si utiliza más de un panel, explique cada uno de ellos; ej.: \emph{Panel superior:} explicación del panel superior. Figura reproducida con permiso de \cite{Hough_etal_BAAA_2020}.}
%\label{Figura}
%\end{figure}

%\subsection{Referencias cruzadas}\label{ref}

%Su artículo debe emplear referencias cruzadas utilizando la herramienta  {\sc bibtex}. Para ello elabore un archivo (como el ejemplo incluido: {\tt bibliografia.bib}) conteniendo las referencias {\sc bibtex} utilizadas en el texto. Incluya el nombre de este archivo en el comando \LaTeX{} de inclusión de bibliografía (\verb|\bibliography{bibliografia}|). 

%Recuerde que la base de datos ADS contiene las entradas de {\sc bibtex}  para todos los artículos. Se puede acceder a ellas mediante el enlace ``{\em Export Citation}''.

%El estilo de las referencias se aplica automáticamente a través del archivo de estilo incluido (baaa.bst). De esta manera, las referencias generadas tendrán la forma co\-rrec\-ta para un autor \citep{hubble_expansion_1929}, dos autores \citep{penzias_cmb_1965,penzias_cmb_II_1965}, tres autores \citep{navarro_NFW_1997} y muchos autores \citep{riess_SN1a_1998}, \citep{Planck_2016}.






%%%%%%%%%%%%%%%%%%%%%%%%%%%%%%%%%%%%%%%%%%%%%%%%%%%%%%%%%%%%%%%%%%%%%%%%%%%%%%
%  ******************* Bibliografía / Bibliography ************************  %
%                                                                            %
%  -Ver en la sección 3 "Bibliografía" para mas información.                 %
%  -Debe usarse BIBTEX.                                                      %
%  -NO MODIFIQUE las líneas de la bibliografía, salvo el nombre del archivo  %
%   BIBTEX con la lista de citas (sin la extensión .BIB).                    %
%                                                                            %
%  -BIBTEX must be used.                                                     %
%  -Please DO NOT modify the following lines, except the name of the BIBTEX  %
%  file (without the .BIB extension).                                       %
%%%%%%%%%%%%%%%%%%%%%%%%%%%%%%%%%%%%%%%%%%%%%%%%%%%%%%%%%%%%%%%%%%%%%%%%%%%%%% 

\bibliographystyle{baaa}
\small
\bibliography{bibliografia}
 
\end{document}
