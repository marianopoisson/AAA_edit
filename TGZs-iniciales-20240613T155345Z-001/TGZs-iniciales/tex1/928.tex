
%%%%%%%%%%%%%%%%%%%%%%%%%%%%%%%%%%%%%%%%%%%%%%%%%%%%%%%%%%%%%%%%%%%%%%%%%%%%%%
%  ************************** AVISO IMPORTANTE **************************    %
%                                                                            %
% Éste es un documento de ayuda para los autores que deseen enviar           %
% trabajos para su consideración en el Boletín de la Asociación Argentina    %
% de Astronomía.                                                             %
%                                                                            %
% Los comentarios en este archivo contienen instrucciones sobre el formato   %
% obligatorio del mismo, que complementan los instructivos web y PDF.        %
% Por favor léalos.                                                          %
%                                                                            %
%  -No borre los comentarios en este archivo.                                %
%  -No puede usarse \newcommand o definiciones personalizadas.               %
%  -SiGMa no acepta artículos con errores de compilación. Antes de enviarlo  %
%   asegúrese que los cuatro pasos de compilación (pdflatex/bibtex/pdflatex/ %
%   pdflatex) no arrojan errores en su terminal. Esta es la causa más        %
%   frecuente de errores de envío. Los mensajes de "warning" en cambio son   %
%   en principio ignorados por SiGMa.                                        %
%                                                                            %
%%%%%%%%%%%%%%%%%%%%%%%%%%%%%%%%%%%%%%%%%%%%%%%%%%%%%%%%%%%%%%%%%%%%%%%%%%%%%%

%%%%%%%%%%%%%%%%%%%%%%%%%%%%%%%%%%%%%%%%%%%%%%%%%%%%%%%%%%%%%%%%%%%%%%%%%%%%%%
%  ************************** IMPORTANT NOTE ******************************  %
%                                                                            %
%  This is a help file for authors who are preparing manuscripts to be       %
%  considered for publication in the Boletín de la Asociación Argentina      %
%  de Astronomía.                                                            %
%                                                                            %
%  The comments in this file give instructions about the manuscripts'        %
%  mandatory format, complementing the instructions distributed in the BAAA  %
%  web and in PDF. Please read them carefully                                %
%                                                                            %
%  -Do not delete the comments in this file.                                 %
%  -Using \newcommand or custom definitions is not allowed.                  %
%  -SiGMa does not accept articles with compilation errors. Before submission%
%   make sure the four compilation steps (pdflatex/bibtex/pdflatex/pdflatex) %
%   do not produce errors in your terminal. This is the most frequent cause  %
%   of submission failure. "Warning" messsages are in principle bypassed     %
%   by SiGMa.                                                                %
%                                                                            % 
%%%%%%%%%%%%%%%%%%%%%%%%%%%%%%%%%%%%%%%%%%%%%%%%%%%%%%%%%%%%%%%%%%%%%%%%%%%%%%

\documentclass[baaa]{baaa}

%%%%%%%%%%%%%%%%%%%%%%%%%%%%%%%%%%%%%%%%%%%%%%%%%%%%%%%%%%%%%%%%%%%%%%%%%%%%%%
%  ******************** Paquetes Latex / Latex Packages *******************  %
%                                                                            %
%  -Por favor NO MODIFIQUE estos comandos.                                   %
%  -Si su editor de texto no codifica en UTF8, modifique el paquete          %
%  'inputenc'.                                                               %
%                                                                            %
%  -Please DO NOT CHANGE these commands.                                     %
%  -If your text editor does not encodes in UTF8, please change the          %
%  'inputec' package                                                         %
%%%%%%%%%%%%%%%%%%%%%%%%%%%%%%%%%%%%%%%%%%%%%%%%%%%%%%%%%%%%%%%%%%%%%%%%%%%%%%
 
\usepackage[pdftex]{hyperref}
\usepackage{subfigure}
\usepackage{natbib}
\usepackage{helvet,soul}
\usepackage[font=small]{caption}

%%%%%%%%%%%%%%%%%%%%%%%%%%%%%%%%%%%%%%%%%%%%%%%%%%%%%%%%%%%%%%%%%%%%%%%%%%%%%%
%  *************************** Idioma / Language **************************  %
%                                                                            %
%  -Ver en la sección 3 "Idioma" para mas información                        %
%  -Seleccione el idioma de su contribución (opción numérica).               %
%  -Todas las partes del documento (titulo, texto, figuras, tablas, etc.)    %
%   DEBEN estar en el mismo idioma.                                          %
%                                                                            %
%  -Select the language of your contribution (numeric option)                %
%  -All parts of the document (title, text, figures, tables, etc.) MUST  be  %
%   in the same language.                                                    %
%                                                                            %
%  0: Castellano / Spanish                                                   %
%  1: Inglés / English                                                       %
%%%%%%%%%%%%%%%%%%%%%%%%%%%%%%%%%%%%%%%%%%%%%%%%%%%%%%%%%%%%%%%%%%%%%%%%%%%%%%

\contriblanguage{1}

%%%%%%%%%%%%%%%%%%%%%%%%%%%%%%%%%%%%%%%%%%%%%%%%%%%%%%%%%%%%%%%%%%%%%%%%%%%%%%
%  *************** Tipo de contribución / Contribution type ***************  %
%                                                                            %
%  -Seleccione el tipo de contribución solicitada (opción numérica).         %
%                                                                            %
%  -Select the requested contribution type (numeric option)                  %
%                                                                            %
%  1: Artículo de investigación / Research article                           %
%  2: Artículo de revisión invitado / Invited review                         %
%  3: Mesa redonda / Round table                                             %
%  4: Artículo invitado  Premio Varsavsky / Invited report Varsavsky Prize   %
%  5: Artículo invitado Premio Sahade / Invited report Sahade Prize          %
%  6: Artículo invitado Premio Sérsic / Invited report Sérsic Prize          %
%%%%%%%%%%%%%%%%%%%%%%%%%%%%%%%%%%%%%%%%%%%%%%%%%%%%%%%%%%%%%%%%%%%%%%%%%%%%%%

\contribtype{1}

%%%%%%%%%%%%%%%%%%%%%%%%%%%%%%%%%%%%%%%%%%%%%%%%%%%%%%%%%%%%%%%%%%%%%%%%%%%%%%
%  ********************* Área temática / Subject area *********************  %
%                                                                            %
%  -Seleccione el área temática de su contribución (opción numérica).        %
%                                                                            %
%  -Select the subject area of your contribution (numeric option)            %
%                                                                            %
%  1 : SH    - Sol y Heliosfera / Sun and Heliosphere                        %
%  2 : SSE   - Sistema Solar y Extrasolares  / Solar and Extrasolar Systems  %
%  3 : AE    - Astrofísica Estelar / Stellar Astrophysics                    %
%  4 : SE    - Sistemas Estelares / Stellar Systems                          %
%  5 : MI    - Medio Interestelar / Interstellar Medium                      %
%  6 : EG    - Estructura Galáctica / Galactic Structure                     %
%  7 : AEC   - Astrofísica Extragaláctica y Cosmología /                      %
%              Extragalactic Astrophysics and Cosmology                      %
%  8 : OCPAE - Objetos Compactos y Procesos de Altas Energías /              %
%              Compact Objetcs and High-Energy Processes                     %
%  9 : ICSA  - Instrumentación y Caracterización de Sitios Astronómicos
%              Instrumentation and Astronomical Site Characterization        %
% 10 : AGE   - Astrometría y Geodesia Espacial
% 11 : ASOC  - Astronomía y Sociedad                                             %
% 12 : O     - Otros
%
%%%%%%%%%%%%%%%%%%%%%%%%%%%%%%%%%%%%%%%%%%%%%%%%%%%%%%%%%%%%%%%%%%%%%%%%%%%%%%

\thematicarea{7}

%%%%%%%%%%%%%%%%%%%%%%%%%%%%%%%%%%%%%%%%%%%%%%%%%%%%%%%%%%%%%%%%%%%%%%%%%%%%%%
%  *************************** Título / Title *****************************  %
%                                                                            %
%  -DEBE estar en minúsculas (salvo la primer letra) y ser conciso.          %
%  -Para dividir un título largo en más líneas, utilizar el corte            %
%   de línea (\\).                                                           %
%                                                                            %
%  -It MUST NOT be capitalized (except for the first letter) and be concise. %
%  -In order to split a long title across two or more lines,                 %
%   please use linebreaks (\\).                                              %
%%%%%%%%%%%%%%%%%%%%%%%%%%%%%%%%%%%%%%%%%%%%%%%%%%%%%%%%%%%%%%%%%%%%%%%%%%%%%%
% Dates
% Only for editors
\received{\ldots}
\accepted{\ldots}




%%%%%%%%%%%%%%%%%%%%%%%%%%%%%%%%%%%%%%%%%%%%%%%%%%%%%%%%%%%%%%%%%%%%%%%%%%%%%%



\title{The globular cluster system of nearby spirals through multi-band surveys: Improving the photometric catalogue}

%%%%%%%%%%%%%%%%%%%%%%%%%%%%%%%%%%%%%%%%%%%%%%%%%%%%%%%%%%%%%%%%%%%%%%%%%%%%%%
%  ******************* Título encabezado / Running title ******************  %
%                                                                            %
%  -Seleccione un título corto para el encabezado de las páginas pares.      %
%                                                                            %
%  -Select a short title to appear in the header of even pages.              %
%%%%%%%%%%%%%%%%%%%%%%%%%%%%%%%%%%%%%%%%%%%%%%%%%%%%%%%%%%%%%%%%%%%%%%%%%%%%%%

\titlerunning{Globular clusters systems of nearby spirals}

%%%%%%%%%%%%%%%%%%%%%%%%%%%%%%%%%%%%%%%%%%%%%%%%%%%%%%%%%%%%%%%%%%%%%%%%%%%%%%
%  ******************* Lista de autores / Authors list ********************  %
%                                                                            %
%  -Ver en la sección 3 "Autores" para mas información                       % 
%  -Los autores DEBEN estar separados por comas, excepto el último que       %
%   se separar con \&.                                                       %
%  -El formato de DEBE ser: S.W. Hawking (iniciales luego apellidos, sin     %
%   comas ni espacios entre las iniciales).                                  %
%                                                                            %
%  -Authors MUST be separated by commas, except the last one that is         %
%   separated using \&.                                                      %
%  -The format MUST be: S.W. Hawking (initials followed by family name,      %
%   avoid commas and blanks between initials).                               %
%%%%%%%%%%%%%%%%%%%%%%%%%%%%%%%%%%%%%%%%%%%%%%%%%%%%%%%%%%%%%%%%%%%%%%%%%%%%%%

\author{
  J.P. Caso\inst{1,2,3}, A.I. Ennis\inst{4,5}, A.L. Chies-Santos\inst{6}, B.J. De B\'ortoli\inst{1,2,3}, R.S. de Souza\inst{7},
  M. Canossa\inst{6}, P. Floriano\inst{6}, E. Godoy\inst{6}, P. Lopes\inst{6}, N.L. Miranda\inst{6} \& C. Bonato\inst{6}
}

\authorrunning{Caso et al.}

%%%%%%%%%%%%%%%%%%%%%%%%%%%%%%%%%%%%%%%%%%%%%%%%%%%%%%%%%%%%%%%%%%%%%%%%%%%%%%
%  **************** E-mail de contacto / Contact e-mail *******************  %
%                                                                            %
%  -Por favor provea UNA ÚNICA dirección de e-mail de contacto.              %
%                                                                            %
%  -Please provide A SINGLE contact e-mail address.                          %
%%%%%%%%%%%%%%%%%%%%%%%%%%%%%%%%%%%%%%%%%%%%%%%%%%%%%%%%%%%%%%%%%%%%%%%%%%%%%%

\contact{jpcaso@fcaglp.unlp.edu.ar}

%%%%%%%%%%%%%%%%%%%%%%%%%%%%%%%%%%%%%%%%%%%%%%%%%%%%%%%%%%%%%%%%%%%%%%%%%%%%%%
%  ********************* Afiliaciones / Affiliations **********************  %
%                                                                            %
%  -La lista de afiliaciones debe seguir el formato especificado en la       %
%   sección 3.4 "Afiliaciones".                                              %
%                                                                            %
%  -The list of affiliations must comply with the format specified in        %          
%   section 3.4 "Afiliaciones".                                              %
%%%%%%%%%%%%%%%%%%%%%%%%%%%%%%%%%%%%%%%%%%%%%%%%%%%%%%%%%%%%%%%%%%%%%%%%%%%%%%

\institute{
  Facultad de Ciencias Astron\'omicas y Geof\'isicas, UNLP, Argentina \and
  Instituto de Astrof\'isica de La Plata, CONICET--UNLP, Argentina \and
  Consejo Nacional de Investigaciones Cient\'ificas y T\'ecnicas, Argentina \and
  Waterloo Centre for Astrophysics, University of Waterloo, Canada \and
  Perimeter Institute for Theoretical Physics, Canada \and
  Instituto de F\'isica, Universidade Federal do Rio Grande do Sul, Brasil \and
  University of Hertfordshire, Reino Unido
}

%%%%%%%%%%%%%%%%%%%%%%%%%%%%%%%%%%%%%%%%%%%%%%%%%%%%%%%%%%%%%%%%%%%%%%%%%%%%%%
%  *************************** Resumen / Summary **************************  %
%                                                                            %
%  -Ver en la sección 3 "Resumen" para mas información                       %
%  -Debe estar escrito en castellano y en inglés.                            %
%  -Debe consistir de un solo párrafo con un máximo de 1500 (mil quinientos) %
%   caracteres, incluyendo espacios.                                         %
%                                                                            %
%  -Must be written in Spanish and in English.                               %
%  -Must consist of a single paragraph with a maximum  of 1500 (one thousand %
%   five hundred) characters, including spaces.                              %
%%%%%%%%%%%%%%%%%%%%%%%%%%%%%%%%%%%%%%%%%%%%%%%%%%%%%%%%%%%%%%%%%%%%%%%%%%%%%%

\resumen{Los relevamientos fotom\'etricos multi-banda permiten estudiar
los sistemas de c\'umulos globulares de un amplio n\'umero de galaxias cercanas en 
forma homog\'enea. El procesamiento de estos datos y la construcci\'on de cat\'alogos
fotom\'etricos, es crucial para determinar una muestra de candidatos a 
partir de m\'etodos estad\'isticos. En el presente trabajo exploramos alternativas 
al procedimiento de reducci\'on aplicado en trabajos previos de este proyecto, 
que permiten mejorar el cat\'alogo de objetos puntuales.}

\abstract{Photometric multi-band surveys are useful for studying the globular
cluster systems for a large number of nearby galaxies in an homogeneous manner. 
The data processing and further catalogues build-up are important in obtaining 
a sample of likely candidates by means of statistical methods. In the present 
work we test some alternatives to the photometrical procedure applied in 
previous studies from this project, to improve the catalogue of point-sources.}

%%%%%%%%%%%%%%%%%%%%%%%%%%%%%%%%%%%%%%%%%%%%%%%%%%%%%%%%%%%%%%%%%%%%%%%%%%%%%%
%                                                                            %
%  Seleccione las palabras clave que describen su contribución. Las mismas   %
%  son obligatorias, y deben tomarse de la lista de la American Astronomical %
%  Society (AAS), que se encuentra en la página web indicada abajo.          %
%                                                                            %
%  Select the keywords that describe your contribution. They are mandatory,  %
%  and must be taken from the list of the American Astronomical Society      %
%  (AAS), which is available at the webpage quoted below.                    %
%                                                                            %
%  https://journals.aas.org/keywords-2013/                                   %
%                                                                            %
%%%%%%%%%%%%%%%%%%%%%%%%%%%%%%%%%%%%%%%%%%%%%%%%%%%%%%%%%%%%%%%%%%%%%%%%%%%%%%

\keywords{ galaxies: star clusters: general --- galaxies: star clusters: individual --- galaxies: stellar content --- galaxies: groups: individual (M81)}

\begin{document}

\maketitle
\section{Introduction}
\label{sec.intro}
The Sloan Digital Sky Survey \citep{yor00} has represented a turning point for modern astronomy,
which is entering an era dominated by large surveys. In particular, photometric multi-band 
surveys, like the Large Synoptic Survey Telescope \citep{ive08} or the Javalambre Photometric 
Local Universe Survey \citep[J-PLUS,][]{cen19}, provide useful data for a large variety of 
astronomical topics with low telescope time. The amount of data released by these surveys allows 
homogeneous studies of large samples of targets, but it also leads to challenges when processing
the data that have to be addressed in an automated manner.

In particular, the study of globular clusters (GCs) in nearby galaxies is relevant for our 
knowledge of galaxy evolution. They are present in the majority of the galaxies with masses 
above $\approx 10^9\,{\rm M_{\odot}}$, spanning different morphologies and environments
\citep[e.g.][]{har15}. The formation and later evolution of GCs, as well as the build-up of 
globular cluster systems (GCSs), are related to the mergers experienced by their host 
galaxies \citep[e.g.][]{kru19,li19}, and their properties provide evidences about the host
galaxy, including the amount of dark matter \citep[e.g.][]{hud14,for18}, the relevance of the 
accretion processes \citep[e.g.][]{deb22}, and the mechanisms that rule the disruptions of
substructure in the inner regions \citep[e.g.][]{cas24}.
This article presents preliminary results from a project devoted to the study of the GCSs
in nearby spirals, through observations from J-PLUS, and continues the 
analysis carried by \citet{chi22} and \citet{cas23} of the M\,81 triplet, chosen as a test 
bench for our project. Its distance \citep[$D\approx 3.6$\,Mpc,][]{tul13}, and the existence 
of previous studies on the GCSs of M\,81 \citep{per95,nan10,nan11,ma17} make it an excellent
target for our purposes. The triplet is conformed by the spiral M\,81, the dominant galaxy in 
the group, the starburst spiral M\,82 and the irregular NGC\,3077. There is evidence of 
interactions, including emission in HI \citep{deb18} and low surface-brightness substructure 
in optical bands \citep{sme20} that seems to connect the galaxies. 

\section{Observations and photometry}

The dataset is conformed by processed images in twelve filters, from $u$ to $z$ (see 
\citealt{cen19} for information about the filters system) for three pointings that cover 
the entire region of the triplet, and span to the South. The images are publicly available 
at the J-PLUS collaboration website\footnote{http://www.j-plus.es/datareleases/data\_release\_dr2}.

\begin{figure}[!t]
  \centering
  \includegraphics[width=\columnwidth]{Fig1.pdf}
  \caption{Stellarity index from S{\sc ource}E{\sc xtractor} against $r_0$ magnitude for
  sources in the field containing M\,81. In the {\it left panel}, the detection of the 
  sources and its stellarity index are obtained from the composite image described in 
  Section\,\ref{sec.phot}. The {\it right panel} is analogue, but the detection and 
  stellarity index are based on the $r$ image. The red circles correspond to sources with 
  proper motions larger than 3 m.a.s. per year, from Gaia DR3, and assumed as Galactic 
  stars. The green triangles correspond to confirmed GCs from the literature.}
  \label{Fig1}
\end{figure}

\subsection{Photometry}
\label{sec.phot}
  In order to increase the signal-to-noise for faint sources, we perform the 
  detection on a composite image, built from the addition of the images 
  corresponding to broad bands $g$, $r$, $i$ and $z$, plus the narrow band 
  images from filters $J0515$,  $J0660$ and $J0861$. Bluer filters are not
  used to build the composite image, since the spectral energy distribution for old GCs drops in flux for wavelengths below 
  $5000\,{\rm \AA}$ \citep[e.g.][]{bro06}. The aperture photometry in the 
  twelve filters is carried using 
  S{\sc ource}E{\sc xtractor} \citep{ber96} in dual mode, with the composite
  image as the reference one, and using a Gaussian filter to classify the 
  sources. The background is measured locally, with a mesh size of 32 pixels, 
  large enough to not affect the photometry. Typically, the mean effective radii 
  of extragalactic GCs is $\approx 3-4$\,pc \citep[e.g.][]{pen08,cas14}. 
  Considering the distance to the triplet and the typical seeing in these 
  images (above 1\,arcsec), GCs can be treated as point-sources. Finally, a 
  selection of point sources is made based on the S{\sc ource}E{\sc xtractor} 
  parameters, ${\rm FWHM}$ and stellarity index, obtained from the composite
  image. This procedure allows us to reach fainter magnitudes, and to improve
  the selection of sources (see Fig.\,\ref{Fig1}). Point sources lacking 
  photometric measurements in more than one filter are excluded, as well as
  those presenting uncertainties above 0.5\,mag in the broad bands (except for
  the $u$ filter). Finally, the definite catalogue is matched with Gaia\,DR3 
  \citep{gai21} to add information on proper motions.
  
\subsection{Photometric corrections}
  The zero points for each filter are calculated from the crossmatch of the 
  point sources with the J-PLUS photometric catalogue, for the corresponding
  aperture radius. Besides, aperture corrections are obtained from several 
  bright and isolated stars in different sections of the fields. In contrast
  to what was performed in \citet{chi22}, the extinction corrections are 
  calculated for each source with a $E_{B-V}$ map built from the extinction 
  calculators of the NASA/IPAC Infrared Science 
Archive\footnote{https://irsa.ipac.caltech.edu/applications/DUST/},
  and used to correct the photometry of the extinction coefficients 
  for the J-PLUS filters from \citet{lop19}. Figure\,\ref{Fig2} shows
  the map with the gradient of colours representing increasing values of 
  reddening, from 0.04\,mag (light yellow) to 0.14\,mag (dark blue). It 
  should be emphasised that sources embedded in the disk of both spirals 
  could still be affected by their intrinsic extinction.

  \begin{figure}[!t]
  \centering
  \includegraphics[width=\columnwidth]{Fig2.pdf}
  \caption{Reddening map ($E_{B-V}$) for the three pointing included in
  this analysis. It is built from the extinction calculators from the 
  NASA/IPAC Infrared Science Archive and the extinction coefficients for
  the J-PLUS filters. The colour gradient moves towards larger $E_{B-V}$
  values, from light yellow to dark blue.}
  \label{Fig2}
\end{figure}

\subsection{Exclusion of contaminants}
The data publicly available from Gaia\,DR3 are useful to identify Galactic 
stars in the photometric catalogue. This is particularly helpful for
points sources brighter than $g=20.5$\,mag, for which more than 50 
per-cent have proper motion measurements ($\mu$, and 90 per-cent if the 
threshold is set at $g=19$\,mag). In \citet{chi22}, the criterion was to 
reject all the sources with $\mu > 3.6\,{\rm mas\,yr^{-1}}$, based on 
the proper motion measurements of confirmed GCs from the literature. In 
Fig.\,\ref{Fig3} the proper motion measurements for confirmed GCs of M\,81 
from the literature, and three times their corresponding uncertainties
are shown . It is noticeable that, although many GCs present non-negligible 
proper motions, the vast majority of them do not surpass the 
$3\cdot \sigma$ value. A few GCs from \citet{per95} exceed this limit, 
but they lack ACS photometry confirming their extended nature, and 
their flux excess from Gaia (i.e., the ratio between the fluxes derived from
the spectra of the object and its point-spread function in the astrometric CCD) 
range $1.15-1.25$, which is expected for 
point sources. Hence, our rejection criterion is $\mu > 3\cdot \sigma_{\mu}$.

Finally, radial velocities from SDSS\,DR16 \citep{ahu20} are included,
discarding those objects with heliocentric velocities $>1000\,{\rm km\,s^{-1}}$, 
based on the heliocentric velocities of the galaxies of the triplet.
The resulting catalogue has $\approx 11,900$ objects.


  \begin{figure}[!t]
  \centering
  \includegraphics[width=\columnwidth]{Fig3.pdf}
  \caption{Proper motions from Gaia\,DR3 ($\mu$) and three times their
  uncertainties, for confirmed GCs of M\,81. The red line represents the
  identity relation.}
  \label{Fig3}
\end{figure}

\section{Conclusion}
 A better segregation of point-like sources in our photometry, plus more a restrictive
 criterion for proper motion measurements, produce a final catalogue with fewer 
 contaminants. Although it is limited to bright sources, the spectroscopic survey 
 from SDSS\,DR16 is also helpful for this goal. This procedure is intended to provide
 an improved input catalogue for statistical methods, as the one applied in \citet{chi22},
 leading to a better selection of GC candidates. 


\begin{acknowledgement}
This work was funded with grants from Consejo
Nacional de Investigaciones Científicas y Técnicas de la República
Argentina, Agencia Nacional de Promoción Científica y Tecnológica,
and Universidad Nacional de La Plata (Argentina). This research 
was supported
in part by Perimeter Institute for Theoretical Physics. Research
at Perimeter Institute is supported by the Government of Canada
through the Department of Innovation, Science and Economic 
Development and by the Province of Ontario through the Ministry of
Research and Innovation. ACS acknowledges funding from the Conselho Nacional de Desenvolvimento Científico e Tecnológico (CNPq) and the Rio Grande
do Sul Research Foundation (FAPERGS) through grants CNPq-
403580/2016-1, CNPq-11153/2018-6, PqG/FAPERGS-17/2551-
0001, FAPERGS/CAPES 19/2551-0000696-9, L’Oréal UNESCO
ABC Para Mulheres na Ciência and the Chinese Academy of Sci-
ences (CAS) President’s International Fellowship Initiative (PIFI)
through grant 2021VMC0005
\end{acknowledgement}

%%%%%%%%%%%%%%%%%%%%%%%%%%%%%%%%%%%%%%%%%%%%%%%%%%%%%%%%%%%%%%%%%%%%%%%%%%%%%%
%  ******************* Bibliografía / Bibliography ************************  %
%                                                                            %
%  -Ver en la sección 3 "Bibliografía" para mas información.                 %
%  -Debe usarse BIBTEX.                                                      %
%  -NO MODIFIQUE las líneas de la bibliografía, salvo el nombre del archivo  %
%   BIBTEX con la lista de citas (sin la extensión .BIB).                    %
%                                                                            %
%  -BIBTEX must be used.                                                     %
%  -Please DO NOT modify the following lines, except the name of the BIBTEX  %
%  file (without the .BIB extension).                                       %
%%%%%%%%%%%%%%%%%%%%%%%%%%%%%%%%%%%%%%%%%%%%%%%%%%%%%%%%%%%%%%%%%%%%%%%%%%%%%% 

\bibliographystyle{baaa}
\small
\bibliography{biblio}
 
\end{document}
