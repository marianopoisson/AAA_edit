
%%%%%%%%%%%%%%%%%%%%%%%%%%%%%%%%%%%%%%%%%%%%%%%%%%%%%%%%%%%%%%%%%%%%%%%%%%%%%%
%  ************************** AVISO IMPORTANTE **************************    %
%                                                                            %
% Éste es un documento de ayuda para los autores que deseen enviar           %
% trabajos para su consideración en el Boletín de la Asociación Argentina    %
% de Astronomía.                                                             %
%                                                                            %
% Los comentarios en este archivo contienen instrucciones sobre el formato   %
% obligatorio del mismo, que complementan los instructivos web y PDF.        %
% Por favor léalos.                                                          %
%                                                                            %
%  -No borre los comentarios en este archivo.                                %
%  -No puede usarse \newcommand o definiciones personalizadas.               %
%  -SiGMa no acepta artículos con errores de compilación. Antes de enviarlo  %
%   asegúrese que los cuatro pasos de compilación (pdflatex/bibtex/pdflatex/ %
%   pdflatex) no arrojan errores en su terminal. Esta es la causa más        %
%   frecuente de errores de envío. Los mensajes de "warning" en cambio son   %
%   en principio ignorados por SiGMa.                                        %
%                                                                            %
%%%%%%%%%%%%%%%%%%%%%%%%%%%%%%%%%%%%%%%%%%%%%%%%%%%%%%%%%%%%%%%%%%%%%%%%%%%%%%

%%%%%%%%%%%%%%%%%%%%%%%%%%%%%%%%%%%%%%%%%%%%%%%%%%%%%%%%%%%%%%%%%%%%%%%%%%%%%%
%  ************************** IMPORTANT NOTE ******************************  %
%                                                                            %
%  This is a help file for authors who are preparing manuscripts to be       %
%  considered for publication in the Boletín de la Asociación Argentina      %
%  de Astronomía.                                                            %
%                                                                            %
%  The comments in this file give instructions about the manuscripts'        %
%  mandatory format, complementing the instructions distributed in the BAAA  %
%  web and in PDF. Please read them carefully                                %
%                                                                            %
%  -Do not delete the comments in this file.                                 %
%  -Using \newcommand or custom definitions is not allowed.                  %
%  -SiGMa does not accept articles with compilation errors. Before submission%
%   make sure the four compilation steps (pdflatex/bibtex/pdflatex/pdflatex) %
%   do not produce errors in your terminal. This is the most frequent cause  %
%   of submission failure. "Warning" messsages are in principle bypassed     %
%   by SiGMa.                                                                %
%                                                                            % 
%%%%%%%%%%%%%%%%%%%%%%%%%%%%%%%%%%%%%%%%%%%%%%%%%%%%%%%%%%%%%%%%%%%%%%%%%%%%%%

\documentclass[baaa]{baaa}

%%%%%%%%%%%%%%%%%%%%%%%%%%%%%%%%%%%%%%%%%%%%%%%%%%%%%%%%%%%%%%%%%%%%%%%%%%%%%%
%  ******************** Paquetes Latex / Latex Packages *******************  %
%                                                                            %
%  -Por favor NO MODIFIQUE estos comandos.                                   %
%  -Si su editor de texto no codifica en UTF8, modifique el paquete          %
%  'inputenc'.                                                               %
%                                                                            %
%  -Please DO NOT CHANGE these commands.                                     %
%  -If your text editor does not encodes in UTF8, please change the          %
%  'inputec' package                                                         %
%%%%%%%%%%%%%%%%%%%%%%%%%%%%%%%%%%%%%%%%%%%%%%%%%%%%%%%%%%%%%%%%%%%%%%%%%%%%%%
 
\usepackage[pdftex]{hyperref}
\usepackage{subfigure}
\usepackage{natbib}
\usepackage{helvet,soul}
\usepackage[font=small]{caption}

%%%%%%%%%%%%%%%%%%%%%%%%%%%%%%%%%%%%%%%%%%%%%%%%%%%%%%%%%%%%%%%%%%%%%%%%%%%%%%
%  *************************** Idioma / Language **************************  %
%                                                                            %
%  -Ver en la sección 3 "Idioma" para mas información                        %
%  -Seleccione el idioma de su contribución (opción numérica).               %
%  -Todas las partes del documento (titulo, texto, figuras, tablas, etc.)    %
%   DEBEN estar en el mismo idioma.                                          %
%                                                                            %
%  -Select the language of your contribution (numeric option)                %
%  -All parts of the document (title, text, figures, tables, etc.) MUST  be  %
%   in the same language.                                                    %
%                                                                            %
%  0: Castellano / Spanish                                                   %
%  1: Inglés / English                                                       %
%%%%%%%%%%%%%%%%%%%%%%%%%%%%%%%%%%%%%%%%%%%%%%%%%%%%%%%%%%%%%%%%%%%%%%%%%%%%%%

\contriblanguage{1}

%%%%%%%%%%%%%%%%%%%%%%%%%%%%%%%%%%%%%%%%%%%%%%%%%%%%%%%%%%%%%%%%%%%%%%%%%%%%%%
%  *************** Tipo de contribución / Contribution type ***************  %
%                                                                            %
%  -Seleccione el tipo de contribución solicitada (opción numérica).         %
%                                                                            %
%  -Select the requested contribution type (numeric option)                  %
%                                                                            %
%  1: Artículo de investigación / Research article                           %
%  2: Artículo de revisión invitado / Invited review                         %
%  3: Mesa redonda / Round table                                             %
%  4: Artículo invitado  Premio Varsavsky / Invited report Varsavsky Prize   %
%  5: Artículo invitado Premio Sahade / Invited report Sahade Prize          %
%  6: Artículo invitado Premio Sérsic / Invited report Sérsic Prize          %
%%%%%%%%%%%%%%%%%%%%%%%%%%%%%%%%%%%%%%%%%%%%%%%%%%%%%%%%%%%%%%%%%%%%%%%%%%%%%%

\contribtype{1}

%%%%%%%%%%%%%%%%%%%%%%%%%%%%%%%%%%%%%%%%%%%%%%%%%%%%%%%%%%%%%%%%%%%%%%%%%%%%%%
%  ********************* Área temática / Subject area *********************  %
%                                                                            %
%  -Seleccione el área temática de su contribución (opción numérica).        %
%                                                                            %
%  -Select the subject area of your contribution (numeric option)            %
%                                                                            %
%  1 : SH    - Sol y Heliosfera / Sun and Heliosphere                        %
%  2 : SSE   - Sistema Solar y Extrasolares  / Solar and Extrasolar Systems  %
%  3 : AE    - Astrofísica Estelar / Stellar Astrophysics                    %
%  4 : SE    - Sistemas Estelares / Stellar Systems                          %
%  5 : MI    - Medio Interestelar / Interstellar Medium                      %
%  6 : EG    - Estructura Galáctica / Galactic Structure                     %
%  7 : AEC   - Astrofísica Extragaláctica y Cosmología /                      %
%              Extragalactic Astrophysics and Cosmology                      %
%  8 : OCPAE - Objetos Compactos y Procesos de Altas Energías /              %
%              Compact Objetcs and High-Energy Processes                     %
%  9 : ICSA  - Instrumentación y Caracterización de Sitios Astronómicos
%              Instrumentation and Astronomical Site Characterization        %
% 10 : AGE   - Astrometría y Geodesia Espacial
% 11 : ASOC  - Astronomía y Sociedad                                             %
% 12 : O     - Otros
%
%%%%%%%%%%%%%%%%%%%%%%%%%%%%%%%%%%%%%%%%%%%%%%%%%%%%%%%%%%%%%%%%%%%%%%%%%%%%%%

\thematicarea{8}

%%%%%%%%%%%%%%%%%%%%%%%%%%%%%%%%%%%%%%%%%%%%%%%%%%%%%%%%%%%%%%%%%%%%%%%%%%%%%%
%  *************************** Título / Title *****************************  %
%                                                                            %
%  -DEBE estar en minúsculas (salvo la primer letra) y ser conciso.          %
%  -Para dividir un título largo en más líneas, utilizar el corte            %
%   de línea (\\).                                                           %
%                                                                            %
%  -It MUST NOT be capitalized (except for the first letter) and be concise. %
%  -In order to split a long title across two or more lines,                 %
%   please use linebreaks (\\).                                              %
%%%%%%%%%%%%%%%%%%%%%%%%%%%%%%%%%%%%%%%%%%%%%%%%%%%%%%%%%%%%%%%%%%%%%%%%%%%%%%
% Dates
% Only for editors
\received{\ldots}
\accepted{\ldots}




%%%%%%%%%%%%%%%%%%%%%%%%%%%%%%%%%%%%%%%%%%%%%%%%%%%%%%%%%%%%%%%%%%%%%%%%%%%%%%



\title{Neutrino production in the cores of active galaxies}

%%%%%%%%%%%%%%%%%%%%%%%%%%%%%%%%%%%%%%%%%%%%%%%%%%%%%%%%%%%%%%%%%%%%%%%%%%%%%%
%  ******************* Título encabezado / Running title ******************  %
%                                                                            %
%  -Seleccione un título corto para el encabezado de las páginas pares.      %
%                                                                            %
%  -Select a short title to appear in the header of even pages.              %
%%%%%%%%%%%%%%%%%%%%%%%%%%%%%%%%%%%%%%%%%%%%%%%%%%%%%%%%%%%%%%%%%%%%%%%%%%%%%%

\titlerunning{Neutrino production in the cores of active galaxies}

%%%%%%%%%%%%%%%%%%%%%%%%%%%%%%%%%%%%%%%%%%%%%%%%%%%%%%%%%%%%%%%%%%%%%%%%%%%%%%
%  ******************* Lista de autores / Authors list ********************  %
%                                                                            %
%  -Ver en la sección 3 "Autores" para mas información                       % 
%  -Los autores DEBEN estar separados por comas, excepto el último que       %
%   se separar con \&.                                                       %
%  -El formato de DEBE ser: S.W. Hawking (iniciales luego apellidos, sin     %
%   comas ni espacios entre las iniciales).                                  %
%                                                                            %
%  -Authors MUST be separated by commas, except the last one that is         %
%   separated using \&.                                                      %
%  -The format MUST be: S.W. Hawking (initials followed by family name,      %
%   avoid commas and blanks between initials).                               %
%%%%%%%%%%%%%%%%%%%%%%%%%%%%%%%%%%%%%%%%%%%%%%%%%%%%%%%%%%%%%%%%%%%%%%%%%%%%%%

\author{A.M. Carulli \inst{1,2}, M.M. Reynoso \inst{1,2} \& L.P. Duvidovich \inst{1,2}
}

\authorrunning{Carulli et al.}

%%%%%%%%%%%%%%%%%%%%%%%%%%%%%%%%%%%%%%%%%%%%%%%%%%%%%%%%%%%%%%%%%%%%%%%%%%%%%%
%  **************** E-mail de contacto / Contact e-mail *******************  %
%                                                                            %
%  -Por favor provea UNA ÚNICA dirección de e-mail de contacto.              %
%                                                                            %
%  -Please provide A SINGLE contact e-mail address.                          %
%%%%%%%%%%%%%%%%%%%%%%%%%%%%%%%%%%%%%%%%%%%%%%%%%%%%%%%%%%%%%%%%%%%%%%%%%%%%%%

\contact{amcarulli@mdp.edu.ar}

%%%%%%%%%%%%%%%%%%%%%%%%%%%%%%%%%%%%%%%%%%%%%%%%%%%%%%%%%%%%%%%%%%%%%%%%%%%%%%
%  ********************* Afiliaciones / Affiliations **********************  %
%                                                                            %
%  -La lista de afiliaciones debe seguir el formato especificado en la       %
%   sección 3.4 "Afiliaciones".                                              %
%                                                                            %
%  -The list of affiliations must comply with the format specified in        %          
%   section 3.4 "Afiliaciones".                                              %
%%%%%%%%%%%%%%%%%%%%%%%%%%%%%%%%%%%%%%%%%%%%%%%%%%%%%%%%%%%%%%%%%%%%%%%%%%%%%%

\institute{
Instituto de Investigaciones F{\'\i}sicas de Mar del Plata, CONICET--UNMdP, Argentina
\and
Departamento de F{\'\i}sica, Facultad de Ciencias Exactas y Naturales, UNMdP, Argentina
}

%%%%%%%%%%%%%%%%%%%%%%%%%%%%%%%%%%%%%%%%%%%%%%%%%%%%%%%%%%%%%%%%%%%%%%%%%%%%%%
%  *************************** Resumen / Summary **************************  %
%                                                                            %
%  -Ver en la sección 3 "Resumen" para mas información                       %
%  -Debe estar escrito en castellano y en inglés.                            %
%  -Debe consistir de un solo párrafo con un máximo de 1500 (mil quinientos) %
%   caracteres, incluyendo espacios.                                         %
%                                                                            %
%  -Must be written in Spanish and in English.                               %
%  -Must consist of a single paragraph with a maximum  of 1500 (one thousand %
%   five hundred) characters, including spaces.                              %
%%%%%%%%%%%%%%%%%%%%%%%%%%%%%%%%%%%%%%%%%%%%%%%%%%%%%%%%%%%%%%%%%%%%%%%%%%%%%%

\resumen{Los núcleos de galaxias activas, como NGC 1068, pueden ser fuentes de neutrinos de alta energía y son opacos a los rayos gamma de alta energía. Los parámetros que caracterizan las regiones de aceleración que conducen a la producción de neutrinos no están claros. En este trabajo presentamos un modelo de producción de neutrinos en chorros que emergen de agujeros negros de masa estelar que están embebidos en el disco de acreción de núcleos galácticos activos. Los resultados obtenidos son consistentes con el flujo electromagnético observado en bandas de diferentes longitudes de ondas. Analizamos las consecuencias de la variación de parámetros como el campo magnético de la región de producción de neutrinos, su tamaño y la influencia del entorno circundante.}

\abstract{The cores of active galaxies, such as NGC 1068, can be sources of high-energy neutrinos, and are opaque to high-energy gamma rays. The parameters that characterize the acceleration regions leading to neutrino production are not clear. In this work we present a model of neutrino production in jets emerging from stellar-mass black holes that are embedded in the accretion disk of active galactic nuclei. The results obtained are consistent with the electromagnetic flux observed in different wavebands. We analyze the consequences of varying parameters such as the magnetic field of the neutrino production region, its size, and the influence of the surrounding environment.}

%%%%%%%%%%%%%%%%%%%%%%%%%%%%%%%%%%%%%%%%%%%%%%%%%%%%%%%%%%%%%%%%%%%%%%%%%%%%%%
%                                                                            %
%  Seleccione las palabras clave que describen su contribución. Las mismas   %
%  son obligatorias, y deben tomarse de la lista de la American Astronomical %
%  Society (AAS), que se encuentra en la página web indicada abajo.          %
%                                                                            %
%  Select the keywords that describe your contribution. They are mandatory,  %
%  and must be taken from the list of the American Astronomical Society      %
%  (AAS), which is available at the webpage quoted below.                    %
%                                                                            %
%  https://journals.aas.org/keywords-2013/                                   %
%                                                                            %
%%%%%%%%%%%%%%%%%%%%%%%%%%%%%%%%%%%%%%%%%%%%%%%%%%%%%%%%%%%%%%%%%%%%%%%%%%%%%%

\keywords{neutrinos --- astroparticle physics --- galaxies: active
}

\begin{document}

\maketitle
\section{Introduction}\label{S_intro}
The IceCube collaboration reported neutrino observations from NGC 1068 (\citealt{Aartsen2020}), a Seyfert galaxy located at $12.7 ~$Mpc from Earth. The mechanism that explains the origin of these neutrinos is unclear. This source has been observed in different wavelenghts. The inner regions were observed by {\sl ALMA} in the near-infrared to radio (\citealt{GarciaBurillo2019}). There are also radio observations of the outer regions ($\sim 100 ~ \mathrm{pc}$) (\citealt{Sajina2011}). Furthermore, gamma rays can be observed up to energies of $100 ~\rm{GeV}$ (\citealt{Ajello2017,Acciari2019,Abdollahi2020}). Many efforts have been made to understand the nonthermal emission from radio-quiet active galactic nuclei (AGN), such as hot coronae models (\citealt{Murase2020,Eichmann2022}), hot accretion flows (\citealt{Gutierrez2021}) and accretion shocks (\citealt{Inoue2020}), among others. The problem arises from the connection between the production of neutrinos and gamma rays, since the processes that lead to the production of neutrinos must not have an associated gamma ray counterpart that violates the available observational data.

In this work, we show preliminary results of a jet model emerging from stellar-mass black holes (sBH) embedded in AGN accretion disks (see Fig.\ref{fig:Figura}). We apply this model following the idea proposed by \cite{Tagawa2023}, which consists on taking sBHs as sources of the emission detected in AGNs, and here we use steady-state transport equations to describe the populations electrons and protons accelerated by inner shocks in the jets of sBHs. This approach allows for a neutrino production scenario and strong suppression of the associated gamma rays.


\section{Model}\label{S_model}

\begin{figure}[!t]
\centering
\includegraphics[width=0.9\linewidth]{dibujo.png}
\caption{Schematic view of the sBH prototype adopted in our model. We show the point of injection $z_j$ and the opening angle of the jet $\theta_j$.}
\label{fig:Figura}
\end{figure}

% Parameters------------------------------------------------------------------------  
\begin{table*} 
\centering
\caption{Input and derived parameters.}
\begin{tabular}{lcccc}
\hline\hline\noalign{\smallskip}
\ \ Parameters & \ \ \ \  Description & \ \ \ \  s1 & \ \ \ \  s2 & \ \ \ \ s3 \\

\multicolumn4{c}{\textit{\rm}} \\

\hline\noalign{\smallskip}
  \ \ $L_{\rm j} \ ({\rm erg \  s^{-1}})$& jet luminosity & $2\times 10^{42}$ & $2 \times 10^{42}$  &  $2\times 10^{42}$\\  
  \ \ $\Gamma \ $ & Lorentz factor & $8$ & $4$ & $4$ \\    
\ \ $\eta$ & acceleration efficiency & $0.001$ & $0.01$ & $0.01$\\   
   \ \ $q_{\rm rel}$& ratio $(L_{\rm e}+L_{\rm p})/L_{\rm {\rm k}}$ & $0.07$ & $0.1$ & $0.2$ \\
    \ \ $a$ & ratio $L_{\rm p}/L_{{\rm e}}$ & $10$ & $1$ & $3$ \\
 \ \ $\epsilon_{\rm B}$ & magnetic field energy fraction & $0.1$& $0.05$ & $0.3$ \\
    \ \ $\Delta z \ ({\rm cm})$ &emission size & $7.8 \times 10^8$& $1.9\times 10^8$ & $1.9\times 10^8$ \\
       \ \ $B \ ({ \rm G})$& magnetic field & $1.6\times 10^{5}$ & $2.5\times 10^{6}$ & $1.9 \times 10^{7}$\\
   \ \ $ n_{\rm p, cold} \ ({\rm cm}^{-3})$ & density of cold protons &  $1.9 \times 10^{14}$&$1.2\times 10^{16}$ & $1.2 \times 10^{16}$\\
   \ \ $\alpha$ & injection index &  $2.6$&$2.2$ & $2.2$\\
  \hline
\label{tabla}
       \end{tabular} 
\end{table*} 


In order to model the emission produced in a jet emerging from a sBH, we consider an homogeneous region within it. The characteristic size of the emission region is $\Delta z_{\rm j} \, \approx \, z_{\rm j} \, \theta_{\rm j}$, where $z_{\rm j}$ is the distance at which we place the injection zone and $\theta_{\rm j}$ is the opening angle of the jet.

The volume of the region is taken as in \cite{Denton2018}:
\begin{equation}
\Delta V \, = \, 4 \, \pi \, (1 - \cos{\theta_{{\rm j}}}) \, z_{{\rm j}}^{2} \, c \, t_{\rm v} \, \Gamma_{\rm j} ,
\end{equation}
where  the variability timescale of the jet is $t_{\rm v}  =  10^{-3} \, {\rm s}$.

The luminosity of the jet is taken as $L_{\rm j} \, = \, \eta_j \, \Dot{m} \, c^2 \,  \approx \, 2\times 10^{42} ~ \rm erg \, s^{-1}$, which corresponds to a super-Eddington accretion rate of $\Dot{m} = 3\times 10^{-4} ~ M_{\odot} \, \rm yr^{-1} \, (\eta_j/0.1)^{-1}$, where $\eta_j = 0.1$ represents the efficiency with which the jet engine converts energy into kinetic energy \citep{Tagawa2023}. Assuming that each sBH accretes for $\sim \,4 ~ \rm Myr$ per AGN phase and the accretion rate is constant, the mass of an sBH increases to $m \, \sim \, 10^3 ~ M_{\odot}$. The fraction of the jet luminosity that is transfered to electrons and protons is
$L_{\rm p} + L_{\rm e} \ = \ q_{\rm rel} \, L_{{\rm j}}$, where the relation between the luminosity of protons and electrons depends on the parameter $a$ as $L_{\rm p}  =  a \, L_{\rm e}$.

The cold proton density in the jet can be obtained as (\citealt{Tagawa2023}):

%\begin{equation}
 %       n_{\rm p} \ = \ \frac{L_j}{2 \, \pi \, \Gamma_{{\rm j}}^{2} \, \theta_{{\rm j}}^{2} \, z_{{\rm j}}^{2} \, m_{\rm p} \, c^3}.
%\end{equation}

\begin{equation}
        n_{\rm p} \ = \left( \frac{2 L_j}{\theta_{{\rm j}}^{2}}\right) \, \frac{1}{4 \, \pi \, \Gamma_{{\rm j}}^{2} \, z_{{\rm j}}^{2} \, m_{\rm p} \, c^3},
\end{equation}
\noindent where the first factor in the right hand side of the equation is the isotropic luminosity (\citealp{Denton2018}).

On the other hand, the magnetic field is obtained as (\citealt{Tagawa2023}):

\begin{equation}
        B_{\rm j} \ = \ \sqrt{8 \, \pi \, \epsilon_{\rm B} \, e_{\rm j}},
\end{equation}
\noindent where $e_{\rm j} \ = \ 2.7 \, n_{\rm p} \, m_{\rm p} \, c^2$ is the internal energy density of the shocked jet, and $\epsilon_{\rm B}$ is the fraction of postshock energy carried by protons and electrons.

In order to calculate the distribution of primary protons and electrons, we solve the steady-state transport equation

\begin{equation} \label{equation}
     \frac{{\rm d}\left[b_{i} \, N_{i}(E_i)\right]}{{\rm d}E} \, + \, \frac{N_{i}(E_i)}{T_{\rm esc}} \ = \ Q_{i}(E_i),
\end{equation}

\noindent where $b_{i} \, \equiv \, dE_i/dt \, = \, -E_i \, t^{-1}_{\rm cool}$ denote the energy losses of the particles and the injection of primary particles can be expressed like

\begin{equation} \label{Qinj}
  Q_{i}(E_i) \ = \ K_{i} \, E_i^{-\alpha} \, \mathrm{e}^{({-E_i}/{E_{{\rm max},i}})} \label{Qi} .
  \end{equation} 

The constant $K_{i}$ is fixed by normalization on the total power injected in electrons and protons

\begin{equation}
L_{i} \ = \ 4 \, \pi \, \Delta V \, \int_{E_{i,\rm min}}^\infty \, {\rm d}E_i \, E\, Q_{i}(E_i).
\end{equation}

Notice that, rather than assuming the primary particle distributions follow a power law in energy, as in \citet{Tagawa2023}, we consider an injection that follows a power law in energy, as described in Eq.(\ref{equation}), and then solve Eq.(\ref{Qinj}).

In order to obtain the maximum energy $E_{{\rm max},i}$, we balance the cooling rates due to radiative processes plus the escape rate with the particle acceleration rate.

Neutrinos are produced when protons collide with cold protons in the jet or when they interact with photons produced via synchrotron of electrons. These interactions produce charged pions that produce neutrinos and muons. We follow \cite{Kelner2006} and \cite{Hümmer2010} to calculate pions produced from proton-proton ($pp$) and proton-gamma ($p\gamma$) interactions respectively. Muons also decay to neutrinos. We follow \cite{Lipari2007} to calculate the neutrino fluxes. On the other hand, neutral pions decay to produce gamma rays.
%, but gamma rays with energies greater than $1 ~ \rm GeV$ are strongly suppresed due to gamma-gamma absorption.
We follow \cite{Kelner2006} and expressions therein to obtain the gamma ray fluxes by $pp$ interactions, and refer to \cite{Kelner2008} to calculate those originated by $p\gamma$ interactions. We consider internal absorption effects that are due to electron synchrotron and IC radiation, and we account for it by multiplying the emission by a factor $(1-\exp(-\tau_{\gamma\gamma}))/\tau_{\gamma\gamma}$, where $\tau_{\gamma\gamma}$ is the optical depth of the process.

\section{Results and discussion}





We consider three sets of parameters, named s1, s2 and s3, for which we vary the jet luminosity, the lorentz factor, the magnetic field, the cold proton target, the size of the zone and the injection index, as shown in Table \ref{tabla}. 

\begin{figure*}[!ht] 
\centering
%%% local settings
\renewcommand{\arraystretch}{0}
\setlength{\tabcolsep}{0pt}

\begin{tabular}{ccc}
\includegraphics[width=0.29\linewidth]{tps1.pdf} &
\includegraphics[width=0.29\linewidth]{tps2.pdf} &
\includegraphics[width=0.29\linewidth]{tps3.pdf} \\
\includegraphics[width=0.29\linewidth]{tes1.pdf} &
\includegraphics[width=0.29\linewidth]{tes2.pdf}  &
\includegraphics[width=0.29\linewidth]{tes3.pdf} \\
\end{tabular}
%
\caption{\textit{Top panels:} Proton cooling rates for the models s1 (\textit{left panel}), s2 (\textit{middle panel}) and s3 (\textit{right panel}). \textit{Bottom panels}: same as above but for electrons.}
\label{fig:pecool}
\end{figure*}



In Fig. \ref{fig:pecool} we show the cooling rates of primary particles as a function of the energy for the models s1, s2 and s3. In the case of electrons, due to the high magnetic field, the synchrotron cooling dominates for all the sets of parameters, but IC is significant as well. In the case of protons, proton-proton (pp) interactions are important for lower energies, whereas synchrotron and p$\gamma$ dominates the cooling rates for sets s2 and s3 at higher energies, and p$\gamma$ dominates in the case of s1. Bohm diffusion was taken for the three sets. In Fig. \ref{fig:Nprimary} we show the resulting proton and electron distributions for the models s1, s2 and s3.

In Fig. \ref{fig:gammas} we show the resulting electromagnatic emission for the three parameter sets. The emission is dominated by the synchrotron of electrons and IC for the three cases, due to the intensity of the magnetic field. It can be seen that for all the cases, the gamma-ray flux suffers a strong suppresion at energies greater than $\sim 1$~GeV for model s1, and $\sim 0.1$~GeV for models s2 and s3.

In Fig. \ref{fig:neutrinos} we show the resulting neutrinos fluxes and compare it with the uncertainties on the spectrum measured by \cite{Aartsen2020}. The contribution to the neutrino flux is dominated by $p\gamma$ interactions for the high energy regime in the three models, whereas $pp$ interactions dominates the emission for energies $\lesssim10^4$GeV for s1 and $\lesssim10^3$GeV for s2 and s3. These contributions can explain the data of the neutrino flux detected by IceCube for NGC 1068 (\citealt{Aartsen2020}), when we consider the set of parameters s3. On the other hand, s2 barely reaches the neutrino flux detected for $10^4$GeV, while the set s1 does not produce enough neutrinos to reach the data.


\begin{figure*} 
\centering
%%% local settings
\renewcommand{\arraystretch}{0}
\setlength{\tabcolsep}{0pt}

\begin{tabular}{cc}
\includegraphics[width=0.4\linewidth]{Ne.pdf} &
\includegraphics[width=0.4\linewidth]{Np.pdf} 
\end{tabular}

\caption{\textit{Left panel:} Electron distributions as a function of the energy, adopting the parameters in table \ref{tabla} for the models s1, s2 and s3. \textit{Right panel:} same as \textit{left panel} but for protons}
\label{fig:Nprimary}
\end{figure*} 

\begin{figure*} 
\centering
%%% local settings
\renewcommand{\arraystretch}{0}
\setlength{\tabcolsep}{0pt}

\begin{tabular}{ccc}
\includegraphics[width=0.33\linewidth]{gammas1.pdf} &
\includegraphics[width=0.33\linewidth]{gammas2.pdf} &
\includegraphics[width=0.33\linewidth]{gammas3.pdf} 
\end{tabular}

\caption{\textit{Left panel:} Photon fluxes as a function of the energy for the model s1. Curves with different colors correspond with different processes as indicated in the legend, as well as the total emission. Dashed lines correspond to the emission with no absorption. \textit{Middle panel:} same as \textit{left panel} but for the model s2. \textit{Right panel:} same as \textit{left panel} but for the model s3.}
\label{fig:gammas}
\end{figure*} 


\begin{figure*} 
\centering
%%% local settings
\renewcommand{\arraystretch}{0}
\setlength{\tabcolsep}{0pt}
\includegraphics[width=0.5\linewidth]{Fnu.pdf}
\caption{Neutrino fluxes as a function of the energy for the models s1, s2 and s3. The blue shaded region represents $1 \sigma$ uncertainty on the spectrum measured by \cite{Aartsen2020}.}
\label{fig:neutrinos}
\end{figure*} 



\section{Conclusion}\label{S_conclusion}

We develop a model that allows to obtain a single object neutrino and photon flux produced by jets embedded in AGNs. In particular, we model neutrinos and gamma-rays emissivities for NGC 1068. The neutrino flux originated due to $p\gamma$ interactions could reach the observed flux reported by \cite{Aartsen2020}, when we consider the set of parameters s3. Future neutrino telescopes such as the next IceCube generation 2 ({\sl IC Gen2})  \citep{SantenIceCubeGen22017} and {\sl KM3NeT} \citep{Aiello2019} will provide further constrains to the model parameters.
On the other hand, the main contribution to the electromagnetic flux is due to synchrotron of electrons and IC. In the gamma-ray band, $p\gamma$ an $pp$ could be significant if internal absortion is not considered. The internal absorption is due to these electrons and supress all the flux for energies greater than $\sim 1$~GeV for model s1, and $\sim 0.1$~GeV for models s2 and s3.






















\begin{acknowledgement}
We thank ANPCyT and Universidad Nacional de Mar del Plata for their financial support through grants PICT 2021-GRF-T1-00725 and EXA1214/24, respectively.
%Los agradecimientos deben agregarse usando el entorno correspondiente (\texttt{acknowledgement}).
\end{acknowledgement}

%%%%%%%%%%%%%%%%%%%%%%%%%%%%%%%%%%%%%%%%%%%%%%%%%%%%%%%%%%%%%%%%%%%%%%%%%%%%%%
%  ******************* Bibliografía / Bibliography ************************  %
%                                                                            %
%  -Ver en la sección 3 "Bibliografía" para mas información.                 %
%  -Debe usarse BIBTEX.                                                      %
%  -NO MODIFIQUE las líneas de la bibliografía, salvo el nombre del archivo  %
%   BIBTEX con la lista de citas (sin la extensión .BIB).                    %
%                                                                            %
%  -BIBTEX must be used.                                                     %
%  -Please DO NOT modify the following lines, except the name of the BIBTEX  %
%  file (without the .BIB extension).                                       %
%%%%%%%%%%%%%%%%%%%%%%%%%%%%%%%%%%%%%%%%%%%%%%%%%%%%%%%%%%%%%%%%%%%%%%%%%%%%%% 

\bibliographystyle{baaa}
\small
\bibliography{bibliografia}
 
\end{document}
