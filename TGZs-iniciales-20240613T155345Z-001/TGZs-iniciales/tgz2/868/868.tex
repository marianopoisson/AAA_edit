
%%%%%%%%%%%%%%%%%%%%%%%%%%%%%%%%%%%%%%%%%%%%%%%%%%%%%%%%%%%%%%%%%%%%%%%%%%%%%%
%  ************************** AVISO IMPORTANTE **************************    %
%                                                                            %
% Éste es un documento de ayuda para los autores que deseen enviar           %
% trabajos para su consideración en el Boletín de la Asociación Argentina    %
% de Astronomía.                                                             %
%                                                                            %
% Los comentarios en este archivo contienen instrucciones sobre el formato   %
% obligatorio del mismo, que complementan los instructivos web y PDF.        %
% Por favor léalos.                                                          %
%                                                                            %
%  -No borre los comentarios en este archivo.                                %
%  -No puede usarse \newcommand o definiciones personalizadas.               %
%  -SiGMa no acepta artículos con errores de compilación. Antes de enviarlo  %
%   asegúrese que los cuatro pasos de compilación (pdflatex/bibtex/pdflatex/ %
%   pdflatex) no arrojan errores en su terminal. Esta es la causa más        %
%   frecuente de errores de envío. Los mensajes de "warning" en cambio son   %
%   en principio ignorados por SiGMa.                                        %
%                                                                            %
%%%%%%%%%%%%%%%%%%%%%%%%%%%%%%%%%%%%%%%%%%%%%%%%%%%%%%%%%%%%%%%%%%%%%%%%%%%%%%

%%%%%%%%%%%%%%%%%%%%%%%%%%%%%%%%%%%%%%%%%%%%%%%%%%%%%%%%%%%%%%%%%%%%%%%%%%%%%%
%  ************************** IMPORTANT NOTE ******************************  %
%                                                                            %
%  This is a help file for authors who are preparing manuscripts to be       %
%  considered for publication in the Boletín de la Asociación Argentina      %
%  de Astronomía.                                                            %
%                                                                            %
%  The comments in this file give instructions about the manuscripts'        %
%  mandatory format, complementing the instructions distributed in the BAAA  %
%  web and in PDF. Please read them carefully                                %
%                                                                            %
%  -Do not delete the comments in this file.                                 %
%  -Using \newcommand or custom definitions is not allowed.                  %
%  -SiGMa does not accept articles with compilation errors. Before submission%
%   make sure the four compilation steps (pdflatex/bibtex/pdflatex/pdflatex) %
%   do not produce errors in your terminal. This is the most frequent cause  %
%   of submission failure. "Warning" messsages are in principle bypassed     %
%   by SiGMa.                                                                %
%                                                                            % 
%%%%%%%%%%%%%%%%%%%%%%%%%%%%%%%%%%%%%%%%%%%%%%%%%%%%%%%%%%%%%%%%%%%%%%%%%%%%%%

\documentclass[baaa]{baaa}

%%%%%%%%%%%%%%%%%%%%%%%%%%%%%%%%%%%%%%%%%%%%%%%%%%%%%%%%%%%%%%%%%%%%%%%%%%%%%%
%  ******************** Paquetes Latex / Latex Packages *******************  %
%                                                                            %
%  -Por favor NO MODIFIQUE estos comandos.                                   %
%  -Si su editor de texto no codifica en UTF8, modifique el paquete          %
%  'inputenc'.                                                               %
%                                                                            %
%  -Please DO NOT CHANGE these commands.                                     %
%  -If your text editor does not encodes in UTF8, please change the          %
%  'inputec' package                                                         %
%%%%%%%%%%%%%%%%%%%%%%%%%%%%%%%%%%%%%%%%%%%%%%%%%%%%%%%%%%%%%%%%%%%%%%%%%%%%%%
 
\usepackage[pdftex]{hyperref}
\usepackage{subfigure}
\usepackage{natbib}
\usepackage{helvet,soul}
\usepackage[font=small]{caption}

%%%%%%%%%%%%%%%%%%%%%%%%%%%%%%%%%%%%%%%%%%%%%%%%%%%%%%%%%%%%%%%%%%%%%%%%%%%%%%
%  *************************** Idioma / Language **************************  %
%                                                                            %
%  -Ver en la sección 3 "Idioma" para mas información                        %
%  -Seleccione el idioma de su contribución (opción numérica).               %
%  -Todas las partes del documento (titulo, texto, figuras, tablas, etc.)    %
%   DEBEN estar en el mismo idioma.                                          %
%                                                                            %
%  -Select the language of your contribution (numeric option)                %
%  -All parts of the document (title, text, figures, tables, etc.) MUST  be  %
%   in the same language.                                                    %
%                                                                            %
%  0: Castellano / Spanish                                                   %
%  1: Inglés / English                                                       %
%%%%%%%%%%%%%%%%%%%%%%%%%%%%%%%%%%%%%%%%%%%%%%%%%%%%%%%%%%%%%%%%%%%%%%%%%%%%%%

\contriblanguage{0}

%%%%%%%%%%%%%%%%%%%%%%%%%%%%%%%%%%%%%%%%%%%%%%%%%%%%%%%%%%%%%%%%%%%%%%%%%%%%%%
%  *************** Tipo de contribución / Contribution type ***************  %
%                                                                            %
%  -Seleccione el tipo de contribución solicitada (opción numérica).         %
%                                                                            %
%  -Select the requested contribution type (numeric option)                  %
%                                                                            %
%  1: Artículo de investigación / Research article                           %
%  2: Artículo de revisión invitado / Invited review                         %
%  3: Mesa redonda / Round table                                             %
%  4: Artículo invitado  Premio Varsavsky / Invited report Varsavsky Prize   %
%  5: Artículo invitado Premio Sahade / Invited report Sahade Prize          %
%  6: Artículo invitado Premio Sérsic / Invited report Sérsic Prize          %
%%%%%%%%%%%%%%%%%%%%%%%%%%%%%%%%%%%%%%%%%%%%%%%%%%%%%%%%%%%%%%%%%%%%%%%%%%%%%%

\contribtype{1}

%%%%%%%%%%%%%%%%%%%%%%%%%%%%%%%%%%%%%%%%%%%%%%%%%%%%%%%%%%%%%%%%%%%%%%%%%%%%%%
%  ********************* Área temática / Subject area *********************  %
%                                                                            %
%  -Seleccione el área temática de su contribución (opción numérica).        %
%                                                                            %
%  -Select the subject area of your contribution (numeric option)            %
%                                                                            %
%  1 : SH    - Sol y Heliosfera / Sun and Heliosphere                        %
%  2 : SSE   - Sistema Solar y Extrasolares  / Solar and Extrasolar Systems  %
%  3 : AE    - Astrofísica Estelar / Stellar Astrophysics                    %
%  4 : SE    - Sistemas Estelares / Stellar Systems                          %
%  5 : MI    - Medio Interestelar / Interstellar Medium                      %
%  6 : EG    - Estructura Galáctica / Galactic Structure                     %
%  7 : AEC   - Astrofísica Extragaláctica y Cosmología /                      %
%              Extragalactic Astrophysics and Cosmology                      %
%  8 : OCPAE - Objetos Compactos y Procesos de Altas Energías /              %
%              Compact Objetcs and High-Energy Processes                     %
%  9 : ICSA  - Instrumentación y Caracterización de Sitios Astronómicos
%              Instrumentation and Astronomical Site Characterization        %
% 10 : AGE   - Astrometría y Geodesia Espacial
% 11 : ASOC  - Astronomía y Sociedad                                             %
% 12 : O     - Otros
%
%%%%%%%%%%%%%%%%%%%%%%%%%%%%%%%%%%%%%%%%%%%%%%%%%%%%%%%%%%%%%%%%%%%%%%%%%%%%%%

\thematicarea{1}

%%%%%%%%%%%%%%%%%%%%%%%%%%%%%%%%%%%%%%%%%%%%%%%%%%%%%%%%%%%%%%%%%%%%%%%%%%%%%%
%  *************************** Título / Title *****************************  %
%                                                                            %
%  -DEBE estar en minúsculas (salvo la primer letra) y ser conciso.          %
%  -Para dividir un título largo en más líneas, utilizar el corte            %
%   de línea (\\).                                                           %
%                                                                            %
%  -It MUST NOT be capitalized (except for the first letter) and be concise. %
%  -In order to split a long title across two or more lines,                 %
%   please use linebreaks (\\).                                              %
%%%%%%%%%%%%%%%%%%%%%%%%%%%%%%%%%%%%%%%%%%%%%%%%%%%%%%%%%%%%%%%%%%%%%%%%%%%%%%
% Dates
% Only for editors
\received{09 February 2024}
\accepted{04 June 2024}




%%%%%%%%%%%%%%%%%%%%%%%%%%%%%%%%%%%%%%%%%%%%%%%%%%%%%%%%%%%%%%%%%%%%%%%%%%%%%%



\title{Efecto Neupert\\ Análisis para fulguraciones del Ciclo Solar 24}

%%%%%%%%%%%%%%%%%%%%%%%%%%%%%%%%%%%%%%%%%%%%%%%%%%%%%%%%%%%%%%%%%%%%%%%%%%%%%%
%  ******************* Título encabezado / Running title ******************  %
%                                                                            %
%  -Seleccione un título corto para el encabezado de las páginas pares.      %
%                                                                            %
%  -Select a short title to appear in the header of even pages.              %
%%%%%%%%%%%%%%%%%%%%%%%%%%%%%%%%%%%%%%%%%%%%%%%%%%%%%%%%%%%%%%%%%%%%%%%%%%%%%%

\titlerunning{Efecto Neupert - Ciclo Solar 24}

%%%%%%%%%%%%%%%%%%%%%%%%%%%%%%%%%%%%%%%%%%%%%%%%%%%%%%%%%%%%%%%%%%%%%%%%%%%%%%
%  ******************* Lista de autores / Authors list ********************  %
%                                                                            %
%  -Ver en la sección 3 "Autores" para mas información                       % 
%  -Los autores DEBEN estar separados por comas, excepto el último que       %
%   se separar con \&.                                                       %
%  -El formato de DEBE ser: S.W. Hawking (iniciales luego apellidos, sin     %
%   comas ni espacios entre las iniciales).                                  %
%                                                                            %
%  -Authors MUST be separated by commas, except the last one that is         %
%   separated using \&.                                                      %
%  -The format MUST be: S.W. Hawking (initials followed by family name,      %
%   avoid commas and blanks between initials).                               %
%%%%%%%%%%%%%%%%%%%%%%%%%%%%%%%%%%%%%%%%%%%%%%%%%%%%%%%%%%%%%%%%%%%%%%%%%%%%%%



\author{G. Cristiani\inst{1} \& C.H. Mandrini\inst{1}}

\authorrunning{Cristiani \& Mandrini}

%%%%%%%%%%%%%%%%%%%%%%%%%%%%%%%%%%%%%%%%%%%%%%%%%%%%%%%%%%%%%%%%%%%%%%%%%%%%%%
%  **************** E-mail de contacto / Contact e-mail *******************  %
%                                                                            %
%  -Por favor provea UNA ÚNICA dirección de e-mail de contacto.              %
%                                                                            %
%  -Please provide A SINGLE contact e-mail address.                          %
%%%%%%%%%%%%%%%%%%%%%%%%%%%%%%%%%%%%%%%%%%%%%%%%%%%%%%%%%%%%%%%%%%%%%%%%%%%%%%

\contact{gcristiani@iafe.uba.ar}

%%%%%%%%%%%%%%%%%%%%%%%%%%%%%%%%%%%%%%%%%%%%%%%%%%%%%%%%%%%%%%%%%%%%%%%%%%%%%%
%  ********************* Afiliaciones / Affiliations **********************  %
%                                                                            %
%  -La lista de afiliaciones debe seguir el formato especificado en la       %
%   sección 3.4 "Afiliaciones".                                              %
%                                                                            %
%  -The list of affiliations must comply with the format specified in        %          
%   section 3.4 "Afiliaciones".                                              %
%%%%%%%%%%%%%%%%%%%%%%%%%%%%%%%%%%%%%%%%%%%%%%%%%%%%%%%%%%%%%%%%%%%%%%%%%%%%%%

\institute{
Instituto de Astronom{\'\i}a y F{\'\i}sica del Espacio, CONICET--UBA, Argentina
}

%%%%%%%%%%%%%%%%%%%%%%%%%%%%%%%%%%%%%%%%%%%%%%%%%%%%%%%%%%%%%%%%%%%%%%%%%%%%%%
%  *************************** Resumen / Summary **************************  %
%                                                                            %
%  -Ver en la sección 3 "Resumen" para mas información                       %
%  -Debe estar escrito en castellano y en inglés.                            %
%  -Debe consistir de un solo párrafo con un máximo de 1500 (mil quinientos) %
%   caracteres, incluyendo espacios.                                         %
%                                                                            %
%  -Must be written in Spanish and in English.                               %
%  -Must consist of a single paragraph with a maximum  of 1500 (one thousand %
%   five hundred) characters, including spaces.                              %
%%%%%%%%%%%%%%%%%%%%%%%%%%%%%%%%%%%%%%%%%%%%%%%%%%%%%%%%%%%%%%%%%%%%%%%%%%%%%%

\resumen{El efecto Neupert representa el resultado emp\'irico que indica que en muchas de las fulguraciones 
solares la derivada temporal de la emisi\'on en rayos X blandos (SXR) muestra concordancia con 
el perfil temporal en rayos X duros (HXR) o en microondas. Esta relaci\'on simple entre las emisiones 
a diferentes frecuencias da sustento a los modelos de fulguraci\'on en los cuales los HXR son 
producidos por bremsstrahlung no t\'ermico de electrones energ\'eticos al termalizarse en la baja corona 
y crom\'osfera, y los SXR corresponden al bremsstrahlung t\'ermico del plasma calentado por los mismos electrones.
 Analizamos este efecto para las fulguraciones m\'as 
energ\'eticas (clasificadas como M o X) pertenecientes al Ciclo Solar 24, usando observaciones en SXR y en 
el rango de las microondas (MW), bajo la hip\'otesis de que los electrones energ\'eticos responsables de la 
emisi\'on en HXR al arribar a la crom\'osfera son los mismos que generan radiaci\'on en microondas por 
emisi\'on girosincrotr\'onica en la baja corona. Observamos, por primera vez, que cuanto mayor es la frecuencia 
en MW considerada, mayor es el acuerdo con el efecto Neupert.  }

\abstract{The Neupert effect empirically states that, for many solar flares, the soft X-ray (SXR) 
time derivative nearly fits the hard X-ray (HXR) or microwave time profiles. This simple relationship supports 
flare models in which the HXR emission is non-thermal bremsstrahlung by accelerated electrons as 
they gradually lose their energy in the lower corona and chromosphere, and the SXR emission is 
thermal bremsstrahlung from plasma heated by the same electrons. We analize this effect for the most 
energetic flares (classified as M or X) during the Solar Cycle 24, using SXR observations and microwave data (MW), 
taking into account that the energetic electrons responsible for HXR emission at chromospheric levels are the same 
that, at low corona levels, produce MW radiation by gyrosynchrotron emission. We observe for the first time 
that, at higher MW frequencies, the Neupert effect works better. }

%%%%%%%%%%%%%%%%%%%%%%%%%%%%%%%%%%%%%%%%%%%%%%%%%%%%%%%%%%%%%%%%%%%%%%%%%%%%%%
%                                                                            %
%  Seleccione las palabras clave que describen su contribución. Las mismas   %
%  son obligatorias, y deben tomarse de la lista de la American Astronomical %
%  Society (AAS), que se encuentra en la página web indicada abajo.          %
%                                                                            %
%  Select the keywords that describe your contribution. They are mandatory,  %
%  and must be taken from the list of the American Astronomical Society      %
%  (AAS), which is available at the webpage quoted below.                    %
%                                                                            %
%  https://journals.aas.org/keywords-2013/                                   %
%                                                                            %
%%%%%%%%%%%%%%%%%%%%%%%%%%%%%%%%%%%%%%%%%%%%%%%%%%%%%%%%%%%%%%%%%%%%%%%%%%%%%%

\keywords{ Sun: flares --- Sun: activity --- Sun: radio radiation --- Sun: X-rays, gamma rays}

\begin{document}

\maketitle

\section{Introducci\'on}\label{S_intro}
Las fulguraciones solares son eventos s\'ubitos de liberaci\'on de energ\'ia ocasionados 
por un proceso de reconexi\'on magn\'etica, en el que se aceleran part\'iculas cargadas 
(principalmente electrones) en l\'aminas de corriente. Estas part\'iculas 
quedan atrapadas en arcos magn\'eticos coronales, donde producen emisi\'on en el rango 
de las microondas por girosincrotr\'on, pero paulatinamente van precipitando 
hacia la crom\'osfera donde, debido a su mayor densidad, se termalizan produciendo 
emisi\'on en rayos X duros por bremsstrahlung de blanco grueso (thick--target bremsstrahlung). Por otro lado se produce 
emisi\'on en rayos X blandos por bremsstrahlung t\'ermico del plasma calentado debido a 
la p\'erdida de energ\'ia de los electrones termalizados. El efecto Neupert \citep{Neupert:1968,Hudson:1991} establece que, 
para la mayor\'ia de las fulguraciones, los flujos integrados temporalmente en el rango 
de las microondas o en rayos X duros ajusta, en forma bastante cercana, las regiones 
crecientes de las curvas de emisi\'on en rayos X blandos o, dicho de otra forma, que el flujo en rayos X 
duros $F_{HXR}(t)$ est\'a relacionado al n\'umero instant\'aneo de electrones mientras que el flujo en rayos X blandos 
$F_{SXR}(t)$ est\'a conectado con la energ\'ia acumulada por estos mismos electrones no t\'ermicos
\begin{equation}F_{HXR}(t)\propto \frac{{\rm d}}{{\rm d}t}F_{SXR}(t)\quad,\end{equation}
donde en nuestro caso usamos el flujo en microondas $F_{MW}$ en lugar de la emisi\'on en rayos X duros. 
Es decir que el efecto Neupert sugiere una relaci\'on causal entre las emisiones t\'ermica 
y no t\'ermica de una fulguraci\'on, relaci\'on que puede explicarse a partir de un mode\-lo de 
fulguraci\'on en que la energ\'ia liberada, en primer t\'ermino, se transforma  en la aceleraci\'on 
de electrones, dichos electrones al precipitarse a la crom\'osfera generan emisi\'on no t\'ermica en rayos X duros y 
calentamiento del plasma cromosf\'erico que luego da lugar al proceso de evaporaci\'on cromosf\'erica, es decir, el llenado de los arcos coronales por un plasma m\'as denso y caliente que origina la emisi\'on t\'ermica en rayos X 
blandos.

 \begin{figure*}[h]
\centering
\includegraphics[width=17.5cm]{fig1.pdf}
\caption{\emph{Panel izquierdo:} Curvas de emisi\'on para la fulguraci\'on del 25/02/2014 00:39 TU, catalogada como X4.9 
seg\'un GOES. El gr\'afico superior muestra la emisi\'on registrada por GOES entre 1 y 8~\AA, el gr\'afico medio 
representa su derivada temporal (agrupada cada 8 segundos para reducir el ruido debido al proceso de derivaci\'on) y el 
gr\'afico inferior reproduce la emisi\'on en 2.695 GHz registrada por RSTN. \emph{Panel central:} Lo mismo que en el anterior 
para el evento del 08/05/2014  09:50 TU, catalogado como M5.3 por GOES, pero la emisi\'on en microondas corresponde a 8.8 GHz. 
\emph{Panel derecho:} Lo mismo que en los anteriores para el evento del 29/03/2014  17:35 TU, catalogado como X1 por GOES, 
pero  la emisi\'on en microondas corresponde a 15.4 GHz. N\'otese en los tres casos la similitud entre las curvas correspondientes 
a la derivada de la emisi\'on en rayos X blandos con las curvas de densidad de flujo en microondas.}
\label{Figura1}
\end{figure*}
Como trabajos previos en que se hace un estudio estad\'istico del efecto Neupert podemos mencionar a 
\citet{Veronigetal:2002} quienes, utilizando datos en rayos X blandos de los Geostationary Operational Evironmental Satellites (GOES) y en rayos X duros del Burst and Transient Source Experiment \citep[BATSE,][]{Harmonetal:2004}, estudiaron fulguraciones de parte del Ciclo Solar 23. Por otro lado, \citet{Effenbergeretal:2017} y \citet{Yuetal:2021} estudiaron el efecto Neupert usando 
observaciones de GOES y datos en rayos X duros del Reuven Ramaty High-Energy Solar Spectroscopic Imager \citep[RHESSI,][]{Linetal:2002}. 
Los primeros se restringieron al estudio de fulguraciones parcialmente ocultas por el limbo (fuentes exclusivamente coronales) 
durante el Ciclo Solar 23, mientras que los \'ultimos analizaron eventos entre 2002 y 2013, lo que abarca parte de los Ciclos 23 y 24.

En este trabajo analizamos la pertinencia o no del efecto Neupert para las fulguraciones del Ciclo Solar 24 
catalogadas como M o X seg\'un la clasificaci\'on de GOES. Se utilizan los datos en microondas con frecuencias 
entre 2.695 y 15.4 GHz obtenidos por la Radio Solar Telescope Network \citep[RSTN,][]{Guidiceetal:1981} y las 
observaciones en rayos X blandos (1 -- 8~{\AA}) de los GOES-14 y GOES-15. La RSTN posee antenas en Sagamore Hill 
(Massachusetts -- EEUU), San Vito (Italia), Learmonth (Australia) y Palehua (Hawaii -- EEUU) lo que le permite disponer 
de una cobertura continua del Sol en frecuencias desde 245 MHz a 15.4 GHz. Este es el primer trabajo estad\'istico de an\'alisis del efecto Neupert usando datos en rayos X blandos junto con datos en microondas en diferentes frecuencias. 


\section{M\'etodo}
Para el an\'alisis del efecto Neupert de un evento en particular, generalmente se comparan las curvas de emisi\'on 
en rayos X duros o microondas con la derivada temporal del flujo en rayos X blandos 
\citep{DennisandZarro:1993,McTiernanetal:1999,Veronig:2003b,NingandCao:2010}. En esta comparaci\'on se analiza la 
conicidencia temporal de los picos de dichas curvas, aunque no necesariamente los tama\~nos relativos de los picos 
se preservan de una curva a otra. Como ejemplo de este tipo de comparaciones, en la Fig. 1 se puede 
apreciar la similitud de las curvas de la derivada de la emisi\'on en rayos X blandos (GOES 1 -- 8~\AA) con la 
densidad de flujo en microondas para las frecuencias de 2.695, 8.8 y 15.4 GHz (RSTN), para algunos eventos seleccionados.


\begin{figure*}[h]
\centering
\includegraphics[width=17.5cm]{fig2.pdf}
\caption{Diagramas de dispersi\'on. En el eje vertical se tiene la intensidad m\'axima del flujo en rayos X blandos 
y en el horizontal el flujo en microondas integrado. La recta en rojo corresponde al ajuste lineal. El panel superior izquierdo 
corresponde a la frecuencia de 2.695 GHz, el panel superior derecho a la frecuencia de 4.995 GHz, el panel inferior 
izquierdo a la frecuencia de 8.8 GHz y el panel inferior derecho a la frecuencia de 15.4 GHz.}
\label{Figura2}
\end{figure*}
Esta similitud en las curvas representa la evidencia gr\'afica del efecto Neupert y tambi\'en el hecho 
de que la emisi\'on en rayos X blandos es de origen t\'ermico mientras que la emisi\'on en microondas se debe 
principalmente a procesos no t\'ermicos (emisi\'on girosincrotr\'on  de una poblaci\'on acelerada de electrones). 
 Sin embargo, la emisi\'on en radio (microondas y frecuencias mayores) no siempre puede asociarse a procesos de 
emisi\'on no t\'ermicos, siendo en ciertas ocasiones el aporte de los procesos t\'ermicos no despreciable en la 
emisi\'on total, o incluso pasando a ser la componente principal de emisi\'on. Un ejemplo de esto lo dan 
Valle Silva et al. [2019]\nocite{ValleSilvaetal:2019} quienes analizan un evento en que la emisi\'on en 
radio en el submilim\'etrico a la frecuencia de 212 GHz representa la componente t\'ermica en el efecto 
Neupert.

En este trabajo de todos los eventos en rayos X blandos observados por GOES en el Ciclo Solar 24 
(m\'as de 8000) nos quedamos s\'olo con los clasificados como M (flujo entre $10^{-5}$ y $10^{-4}$~W/m$^2$) 
o X (flujo mayor a $10^{-4}$~W/m$^2$), es decir los m\'as energ\'eticos (representan alrededor de un 8 \% del total) 
y para los cuales la determinaci\'on  de su contraparte en microondas es m\'as factible. En rigor de verdad, tambi\'en 
se incluyeron unos pocos eventos clasificados como C (flujo entre $10^{-6}$ y $10^{-5}$~W/m$^2$) cuya asociaci\'on 
con eventos en microondas result\'o evidente al momento de analizar las curvas de flujo en microondas para los eventos 
M y X. Esta asociaci\'on se logr\'o en alrededor de un 70\% de los casos, siendo los casos con asociaci\'on fallida 
debidos a que no se observ\'o un incremento significativo en la emisi\'on en radio o a la existencia de vac\'ios 
observacionales en los datos de RSTN. La asociaci\'on de los eventos se realiz\'o en forma manual, es decir, 
observando la emisi\'on en microondas en el rango temporal establecido por cada evento GOES. Al trabajar con fulguraciones  X y M, la mayor\'ia de las veces la asociaci\'on en microondas era inequ\'ivoca; en los casos en que la asociaci\'on no era tan evidente (la mayor\'ia de las veces por observarse una emisi\'on ruidosa) recurrimos al 
criterio de $3\sigma$: consideramos el 
incremento en radio como relevante si el pico de la emisi\'on superaba el valor medio de la emisi\'on en el rango temporal considerado 
m\'as tres veces la desviaci\'on est\'andar. Este criterio se aplic\'o a la emisi\'on en cada frecuencia, teni\'endose en general el 
caso que un evento result\'o detectado en cierta/s frecuencia/s mientras que en otra/s no. 


Una vez establecido nuestro conjunto de eventos procedimos a integrar el flujo 
en radio tomando como per\'iodo de integraci\'on el tiempo que va desde el comienzo hasta el pico del evento en 
rayos X blandos informado por GOES. Para la integraci\'on se utiliz\'o una rutina basada en la suma de trapezoides que adem\'as 
tiene en cuenta la posibilidad de que los datos temporales no se encuentren igualmente espaciados, detalle que es relevante 
para los datos del RSTN, ya que es habitual que las series temporales registradas por este conjunto de radiotelescopios presenten 
falta de regularidad en la toma de datos, aunque nominalmente deber\'ia corresponder a una cadencia de un registro por 
segundo. El flujo integrado se calcul\'o extrayendo el fondo, el cual siempre se consider\'o como el valor m\'as bajo 
de la emisi\'on en microondas en el per\'iodo de integraci\'on correspondiente, y el cual, en la gran mayor\'ia de los casos, 
coincidi\'o con el valor del flujo en el momento inicial del per\'iodo. En definitiva, cada evento de nuestro conjunto 
qued\'o caracterizado por la diferencia del flujo en rayos X blandos entre el inicio y el m\'aximo y el flujo integrado en radio, 
durante el dado intervalo de tiempo, para cada una de las cuatro frecuencias en que el evento fue efectivamente detectado. 
Cabe aclarar que en cada evento determinado no siempre fue posible observar la emisi\'on en las cuatro diferentes frecuencias 
de RSTN, ya sea porque no se observ\'o un incremento efectivo de la emisi\'on o porque no se registraron observaciones 
en alguna de las frecuencias. En consecuencia, el n\'umero de eventos en cada frecuencia difiere de un caso a otro.


\section{Resultados preliminares}

Para cada frecuencia de RSTN hicimos un diagrama de dispersi\'on, los cuales se muestran en la Fig. 2. 
El eje de las ordenadas representa la intensidad del evento en rayos X blandos mientras que el eje de las abscisas da 
el flujo integrado en radio para las cuatro frecuencias analizadas de RSTN. Se hizo 
un ajuste lineal de los datos (trazo rojo), obteni\'endose un coeficiente de correlaci\'on lineal   $r=0.55$ para 
2.695 GHz, $r=0.6$ para 4.995 GHz, $r=0.64$ para 8.8 GHz y $r=0.70$ para 15.4 GHz. Este \'ultimo valor es similar 
al valor obtenido por Veronig [2003], $r=0.71$, en su estudio del efecto Neupert en el que compar\'o intensidad en 
rayos X blandos con la fluencia en rayos X duros.



Si comparamos los valores de correlaci\'on mencionados con los que se obtienen al correlacionar 
el pico en rayos X blandos con los valores pico de la emisi\'on en microondas, podemos notar que estos \'ultimos son 
notoriamente menores ($0.3<r<0.4$). Esto indicar\'ia que la correlaci\'on es debida principalmente a la relaci\'on 
pico en rayos X blandos -- flujo integrado en microondas, como predice el efecto Neupert, y no al hecho que fulguraciones 
con mayo\-res flujos en rayos X blandos tambi\'en tienden a tener contrapartes en microondas m\'as intensas. 
 

\section{Conclusiones y trabajo a futuro}
A partir del an\'alisis del efecto Neupert para los eventos M y X del Ciclo Solar 24 (y unos pocos eventos C), 
u\-san\-do datos en 
rayos X blandos y en microondas, podemos concluir que cuanto mayor es la frecuencia en microondas considerada  
mejor es el acuerdo con lo predicho por el efecto Neupert, resultado que se evidencia con el valor creciente con la 
frecuencia del coeficiente de correlaci\'on lineal. Este estudio estad\'istico del efecto Neupert utilizando diferentes 
frecuencias en microondas es abordado por primera vez en este trabajo, por lo que puede decirse que el resultado 
obtenido es novedoso. El marco te\'orico del efecto Neupert supone que la emisi\'on en 
microondas se debe a electrones energ\'eticos ``espiralando'' alrededor de las l\'ineas de campo magn\'etico de la baja corona 
y produciendo emisi\'on girosincrotr\'on, mientras que la emisi\'on en rayos X blandos se produce como respuesta al 
calentamiento del plasma que originan esos mismos electrones cuando precipitan hacia la crom\'osfera. Que el efecto 
Neupert sea m\'as ostensible a las frecuencias mayores de microondas que a las menores podr\'ia indicar que 
para las frecuencias menores los mecanismos de emisi\'on t\'ermicos son m\'as relevantes, en un sentido estad\'istico, 
que para las frecuencias mayores.

Como trabajo a realizar se podr\'ia tratar de incorporar un mayor n\'umero de eventos C a la estad\'istica, 
tratando de hallar la contraparte en microondas para estos eventos observados en rayos X blandos. Esto implica un volumen 
de trabajo considerable, teniendo en cuenta que los eventos C son mucho m\'as numerosos que los M y los X, aunque 
probablemente un bajo porcentaje de ellos muestren una contraparte claramente detectada en microondas. Por otro lado,  
la clasificaci\'on de los eventos por tipo impulsivo o gradual, la cual para eventos particulares en ocasiones no suele 
ser tan clara, nos permitir\'ia hacer un estudio discriminando por tipo de evento y constatar si esta separaci\'on lleva 
a resultados notoriamente diferentes, o no, en cuanto al an\'alisis del efecto Neupert.

En un trabajo reciente Kazachenko [2023]\nocite{Kazachenko:2023} correlaciona distintos par\'ametros de las 
fulguraciones del Ciclo Solar 24, las cuales cataloga en confinadas y eruptivas, de acuerdo a su asociaci\'on 
con las eyecciones coronales de masa (CMEs): en una fulguraci\'on eruptiva se eyecta plasma hacia el espacio 
interplanetario, el cual es observado posteriormente como una CME en im\'agenes coronogr\'aficas en luz blanca, aunque 
su visibilidad depende en gran medida de la intensidad de la fulguraci\'on; en una fulguraci\'on compacta, de haber plasma 
eyectado este cae nuevamente hacia el Sol y no se produce CME. En dicho trabajo, una de las conclusiones a las que se arriba 
es que las fulguraciones confinadas son más eficientes en la aceleraci\'on de part\'iculas. Esta es otra forma en la que se 
podr\'ian discriminar los eventos para analizar si esta separaci\'on es relevante al momento de evaluar el efecto Neupert. 



%\begin{acknowledgement}
%Los agradecimientos deben agregarse usando el entorno correspondiente (\texttt{acknowledgement}).
%\end{acknowledgement}

%%%%%%%%%%%%%%%%%%%%%%%%%%%%%%%%%%%%%%%%%%%%%%%%%%%%%%%%%%%%%%%%%%%%%%%%%%%%%%
%  ******************* Bibliografía / Bibliography ************************  %
%                                                                            %
%  -Ver en la sección 3 "Bibliografía" para mas información.                 %
%  -Debe usarse BIBTEX.                                                      %
%  -NO MODIFIQUE las líneas de la bibliografía, salvo el nombre del archivo  %
%   BIBTEX con la lista de citas (sin la extensión .BIB).                    %
%                                                                            %
%  -BIBTEX must be used.                                                     %
%  -Please DO NOT modify the following lines, except the name of the BIBTEX  %
%  file (without the .BIB extension).                                       %
%%%%%%%%%%%%%%%%%%%%%%%%%%%%%%%%%%%%%%%%%%%%%%%%%%%%%%%%%%%%%%%%%%%%%%%%%%%%%% 

\bibliographystyle{baaa}
\small
\bibliography{referencias868}
 
\end{document}
